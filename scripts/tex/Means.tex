\section{Means}

\shortdef{Secondary action or event that characterizes \textbf{how} 
the main event happens or is achieved.}

Prototypically a volitional action, though not necessarily \cref{ex:chlorophyll}. 
A volitional \psst{Means} will often modify an intended result, 
though the outcome can be unintended as well \cref{ex:oops}.
\begin{exe}
  \ex Open the door \p{by} turning the knob.
  \ex They retaliated \choices{\p{by} shooting\\\p{with} shootings}.
  \ex\label{ex:oops} The owners destroyed the company \p{by} growing it too fast.
  \ex\label{ex:chlorophyll} Chlorophyll absorbs the light \p{by} transfer of electrons.
\end{exe}

\psst{Means} is similar to \psst{Instrument}, which is used for causally supporting entities 
and is a kind of \psst{Participant}.
See also \psst{Manner}, \psst{Topic}.

Contrast with \psst{Explanation}, which characterizes \textbf{why} 
something happens. I.e., an \psst{Explanation} portrays the secondary event 
as the causal \emph{instigator} of the main event, whereas \psst{Means} 
portrays it merely as a \emph{facilitator}.

\begin{history}
  In v1, \psst{Means} was a subtype of \psst{Instrument}, 
  but with the removal of multiple inheritance for v2, 
  the former was moved directly under \psst{Circumstance} 
  and the latter directly under \psst{Participant}.
\end{history}

