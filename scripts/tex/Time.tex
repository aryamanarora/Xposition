\section{Time}

\shortdef{\textbf{When} something happened or will happen, in relation to an 
explicit or implicit reference time or event.}

\begin{exe}
  \ex We ate \choices{\p{in} the afternoon\\\p{during} the afternoon\\\p{at} 2:00\\\p{on} Friday}.
  \ex\label{ex:at-lunch} Let's talk \choices{\p{at}\\\p{during}} lunch. [compare \cref{ex:over-lunch}]
\end{exe}
For a containing time period or event, \p{during} can be used and is unambiguously \psst{Time}---%
unlike \p{in}, \p{at}, and \p{on}, which can also be locative.\footnote{See \cref{sec:temploc} regarding the 
use of locational metaphors for temporal relations.}
\begin{exe}
  \ex\begin{xlist}
    \ex They will greet us \choices{\p{on}\\\p{upon}} our arrival.
    \ex\label{ex:onXOccasion} I succeeded \p{on} \choices{the fourth attempt\\several occasions}. [contrast \emph{on occasion}, \cref{ex:onOccasion}]
  \end{xlist}
  \ex\label{ex:as-when}\p*{As}{as} meaning `when' (contrast \cref{ex:as-while} under \psst{Circumstance}):\begin{xlist}
    \ex The lights went out \p{as} I opened the door.
    \ex A bee stung me \p{as} I was eating lunch.
  \end{xlist}
  \ex I will finish \p{after} \choices{tomorrow\\lunch\\you (do)}.
  \ex I will finish \p{by} \choices{tomorrow\\lunch}.
  \ex I will contact you \choices{\p{as\_soon\_as}\\once} it's ready.
\end{exe}

The preposition \p{since} is ambiguous:
\begin{exe}
  \ex {} [`after'] I bought a new car---that was \p{since} the breakup. (\psst{Time})
  \ex {} [`ever since'] I have loved you \p{since} the party where we met. (\psst{StartTime}) %\ab{are 9 \& 10 really different?\nss{yes, 10 cannot be paraphrased with `after'}}
  \ex {} [`because'] I'll try not to whistle \p{since} I know that gets on your nerves. (\psst{Explanation})
\end{exe}

Simple \psst{Time} is also used if the reference time is implicit and determined from 
the discourse:
\begin{exe}
  \ex We broke up last year, and I haven't seen her \p{since}. [since we broke up]
\end{exe}

However, \rf{Time}{Interval} is used for adpositions whose complement (object) 
is the amount of time between two reference points:
\begin{exe}
  \ex We left the party \p{after} an hour. [an hour after it started] (\rf{Time}{Interval})
  \ex We left the party an hour \p{ago}. [an hour before now] (\rf{Time}{Interval})
\end{exe}

The preposition \p{over} is also ambiguous:
\begin{exe}
  \ex The deal was negotiated \p{over} (the course of) a year. (\psst{Duration})
  \ex He arrived in town \p{over} the weekend. (\rf{Time}{Duration})
\end{exe}
See discussion under \psst{Duration}.

If the scene role is \psst{Time}, the PP can usually be questioned with \emph{When?}.

\psst{Time} is also used for special constructions for expressing clock times, e.g.~identifying 
a time via an offset:
\begin{exe}
  \ex\begin{xlist}
    \ex The alarm rang at \choices{a quarter \p{after}\\half \p{past}} 8. (\psst{Time})
    \ex\label{ex:quarterTo} The alarm rang at a quarter \p{to} 8. (\rf{Time}{Goal})
    \ex The alarm rang at a quarter \p{of} 8.\footnote{In some dialects, this is an alternate way to express the same meaning as \cref{ex:quarterTo}. 
    It seems that \p{to} and \p{of} construe the same time interval from opposite directions.} (\rf{Time}{Source})
  \end{xlist}
  \ex The alarm rang 15~minutes \p{before} 8. (\psst{Time}) [``15~minutes'' modifies the PP]
\end{exe}

\begin{history}
  In v1, point-like temporal prepositions (\p{at}, \p{on}, \p{in}, \p{as}) 
  were distinguished from displaced temporal prepositions (\p{before}, \p{after}, etc.)\ 
  which present the two times in the relation as unequal. 
  \sst{RelativeTime} inherited from \psst{Time} and was reserved for the 
  displaced temporal prepositions, as well as subclasses \psst{StartTime}, 
  \psst{EndTime}, \sst{DeicticTime}, and \sst{ClockTimeCxn}. 
  
  For v2, \sst{RelativeTime} was merged into \psst{Time}: the distinction 
  was found to be entirely lexical and lacked parallelism with the spatial hierarchy. 
  \sst{ClockTimeCxn} was also merged with \psst{Time}, the usages covered by the former 
  (expressions of clock time like \pex{ten \p{to} seven})
  being exceedingly rare and not very different semantically from 
  prepositions like \p{before}.
  \sst{DeicticTime} became \psst{Interval}.
\end{history}

