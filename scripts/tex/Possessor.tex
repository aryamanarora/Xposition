\section{Possessor}

\shortdef{Animate who \textbf{has} something (the \psst{Possession}) 
which is not part of their body 
or inherent to their identity/character but could, in principle, be taken away.}

Prototypically expressed with the \emph{s-genitive} (\cref{sec:genitives}: \p{'s} and possessive pronouns), 
and \p{of} (the \emph{of-genitive}):

\begin{exe}
{\setlength\multicolsep{0pt}%
\begin{multicols}{2}
  \ex\begin{xlist}
  % of-genitive
  \ex the house \p{of} the Smith family
  \ex the corgis \p{of} Queen Elizabeth
  % s-genitive
  \sn the Smith family\p{'s} house
  \sn Queen Elizabeth\p{'s} corgis
  % \ex the rich \p{'s} money}
  % \ajb{\ex \choices{John \p{'s}\\\p{our}} dog
  % \ex ?the dog \p{of} John
  % \ex the dog \p{of} ours
\end{xlist}
\end{multicols}}
\end{exe}
\psst{Possessor} is not limited to cases of \emph{ownership}, but also includes temporary forms of possession, 
such when something is on loan to or under the control of the possessor:
\begin{exe}
  \ex John\p{'s} hotel room [the room John is staying in as a guest]
  \ex Mary\p{'s} delivery truck [the company truck that Mary drives as an employee]
\end{exe}
A wearer of attire is also in this category:
\begin{exe}
{\setlength\multicolsep{0pt}%
\begin{multicols}{2}
  % of-genitive
  \ex the cloak \p{of} He-Who-Must-Not-Be-Named
  % s-genitive
  \sn He-Who-Must-Not-Be-Named\p{'s} cloak
\end{multicols}}
\ex the cloak \p{on} He-Who-Must-Not-Be-Named (\rf{Possessor}{Locus})
\end{exe}

See \psst{Accompanier}, \psst{Beneficiary}, \psst{OrgRole}.

