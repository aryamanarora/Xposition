\section{Extent}

\shortdef{The size of a path, amount of change, or degree.}

This can be the physical distance traversed or the amount of change on a scale:
\begin{exe}
  \ex We ran \p{for} miles.
  \ex The price shot up \p{by} 10\%.
  \ex an increase \p{of} 10\% (\rf{Extent}{Identity})
\end{exe}
For static distance measurements, see \psst{Direction}.

For scalar \p{as} (see \cref{sec:as-as}), \psst{Extent} serves as the function (and sometimes also the role):
\begin{exe}
  \ex\begin{xlist}
    \ex I helped \p{as} much as I could. (\psst{Extent})
    \ex Your face is \p{as} red as a rose. (\rf{Characteristic}{Extent})
    \ex I stayed \p{as} long as I could. (\rf{Duration}{Extent})
  \end{xlist}
\end{exe}
    
\psst{Extent} also covers degree expressions, such as the following PP idioms:
\begin{exe}\ex\begin{xlist}
  \ex I'm not tired \p{at}\_all.
  \ex The food is mediocre \p{at}\_best.
  \ex You should \p{at}\_least try.
  \ex It is the worst \p{by}\_far.
  \ex We've finished \p{for}\_the\_most\_part.
  \ex It was a success \choices{\p{in}\_every\_respect\\\p{on}\_all\_levels}.
  \ex I hate it when they repeat a song \p{to}\_death.
\end{xlist}\end{exe}
Typically these are licensed by a verb or adjective.


