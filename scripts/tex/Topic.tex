\section{Topic}

\shortdef{Information content or subject matter in communication or cognition, 
or the matter something pertains to.}

A variety of prepositions---including the vast majority of occurrences of \p{about}---can 
mark a \psst{Topic}. The following subclasses warrant \psst{Topic} as the scene role:

\begin{itemize}
  \item \textbf{Communication} scenes: the content or subject matter of 
  speech, writing, art, performance, etc.
  \begin{xexe}
    \ex I \choices{gave a presentation\\spoke} \p{about}/\p{on} politics.
    \ex They wouldn't stop arguing \p{over} the plan.
    \ex I was accused \p{of} treason.
    \ex a picture \p{of} Whistler's mother
    \ex three \choices{copies\\versions} \p{of} the test
    \ex\rf{Topic}{Identity}---see discussion at \psst{Identity}:
      \begin{xlist}
        \ex the topic/issue/question \p{of} semantics
        \ex the idea \p{of} raising money
      \end{xlist}
    \ex The \choices{ratings\\reviews} \p{for} this film are atrocious.
    \ex I did not hazard a guess \p{as\_to} the cause.
  \end{xexe}
  \item \textbf{Cognition} scenes: the content or subject matter of thought and knowledge---belief, opinion,
  decision, learning, study, interest, expertise, skill, etc.
  \begin{exe}
    \ex\begin{xlist}
      \ex Try not to think \p{about} it.
      \ex We took a minute to \choices{think\\ponder} \p{over} the situation.
      \ex I plan \p{on} going again.
      \ex I am focused \p{on} the task at hand.
      \ex There is not enough research \p{on} the effects of global warming.
      \ex She was dumbfounded \p{as\_to} why the police had done that.
      \ex Think \p{of} all the possibilities!
      \ex I have no memory \p{of} the incident.
      \ex I am aware \p{of} the problem.
      \ex You can have your choice \p{of} chicken or fish.
      \ex I disagree \p{with} that statement.
      \ex I am familiar \p{with} this topic.
      \ex Are you interested \p{in} politics?
      \ex I'm confident \p{in} your abilities.
    \end{xlist}
  \ex\label{ex:Activity}\begin{xlist}
    \ex My daughter excels \choices{\p{in}\\\p{at}} sports.
    \ex\label{ex:cookieExpert} I'm \choices{an expert\\talented\\good} \p{at} baking cookies.
    \ex\label{ex:in-activity-Topic} 
      I wouldn't hestitate \p{in} seeing a doctor.\\{} 
      [but see \cref{ex:in-activity-Circumstance} under \psst{Circumstance}, which is syntactically parallel]
    \end{xlist}
  \end{exe}
  \item Relations of \textbf{regard}: the entity, issue, or aspect that the governing 
  predicate pertains to. The relation to the governor may be somewhat loose, 
  skirting the boundary between semantics and information structure.
  \begin{xexe}
    \ex Be reasonable \p{with} your expectations!
    \ex They are transparent \p{with} their fee.
    \ex The discount should apply \p{with} other restaurants too.
    \ex I approached the manager \p{about} the poor service. [implied communication]
    \ex I am a big baby \p{about} needles. [implied cognition]
    %\ex Don't hesitate \p{in} sending us a donation. -- 'hesitate' under cognition
    \ex The owner wouldn't budge \p{on} the price.
    \ex They came through \p{on} all of their promises.
    % they are going to mark you up |on| that feel good premise .
    % Hit or miss |on| the service .
    % They have a great lunch special with your choice |of| meat , chicken , steak , or pork 
    % The best darlington has to offer |in| contemporary sandwicheering .
    \ex She did not do the right thing \p{for} an item that was marked incorrectly.
    \ex I'm fast \p{at} baking cookies. [cf.~\cref{ex:cookieExpert}]
    \ex They have almost anything you could want \choices{\p{when\_it\_comes\_to}\\\p{in\_terms\_of}} spy and surveillance equipment .
  \end{xexe}
\end{itemize}

A few specific governors merit further discussion:

\paragraph{\pex{agree}.}
\begin{xexe}
  \ex Let us agree \p{on} the deal. (\psst{Topic})
  \ex Let us agree \p{to} the deal. (\rf{Topic}{Goal})
\end{xexe}

\paragraph{\pex{answer}, \pex{respond}, etc.}
\begin{exe}
  \ex\rf{Topic}{Goal}:\begin{xlist}
    \ex the answer \p{to} the question
    \ex my response \p{to} your question
  \end{xlist}
\end{exe}

For \w{respond \p{with}} and similar, it depends whether the object is an action, 
a device facilitating communication, or some aspect of transferred information:
\begin{xexe}
  \ex He responded to my kick \p{with} a punch. (\psst{Means})
  \ex He responded to my accusation \p{with} a lawsuit. (\psst{Means})
  \ex He responded to my accusation \p{with} dishonest emails. (\psst{Instrument})
  \ex He responded to my accusation \p{with} falsehoods. (\psst{Topic})
\end{xexe}

\paragraph{\pex{problem \p{with}}, \pex{experience \p{with}}, etc.} 
These are simply \psst{Topic}:
\begin{xexe}
  \ex \choices{There was\\We had} a problem \p{with} mice in the basement.
  \ex I have limited experience \p{with} numerical methods.
  \ex \choices{I had a bad experience\\my bad experience} \p{with} a vampire.
\end{xexe}

See also: \psst{Stimulus}

\begin{history}
  Previously, \lbl{Activity} covered usages such as in \cref{ex:Activity}, 
  but such usages were found to be infrequent and 
  \lbl{Activity} was deemed too narrow.
\end{history}

