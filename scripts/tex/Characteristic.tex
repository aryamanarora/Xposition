\section{Characteristic}

\shortdef{Generalized notion of a part, feature, possession, 
or the contents or composition of something, 
understood with respect to that thing (the \psst{Gestalt}).}

Can be used to construe person-to-person relationships such as kinship, 
whose scene role should be \psst{SocialRel}. 
Labels \psst{Possession}, \psst{PartPortion}, and its subtype \psst{Stuff} 
are defined for some important subclasses.

\psst{Characteristic} applies directly to:
\begin{itemize}
\item	A property value: 
\begin{exe} 
  \ex Adnominal: \rf{Characteristic}{Identity}\begin{xlist}
    \ex a car \p{of} high quality
    \ex a man \p{of} honor
    \ex a business \p{of} that sort [contrast with \psst{Species}, \cref{sec:Species}]
  \end{xlist}
  \ex Secondary predicate adjective: \rf{Characteristic}{Identity}\begin{xlist}
    \ex She described him \p{as} sad.
    \ex He strikes me \p{as} sad.
  \end{xlist}
\end{exe}
\item	Role of a complex framal \psst{Gestalt} that has no obvious decomposition into parts: 
\begin{exe}\ex \begin{xlist}
  \ex the restaurant \p{with} \choices{a convenient location\\an extensive menu}
  \ex a party \p{with} great music
\end{xlist}\end{exe}
\item	That which is located in a container denoted by the governor: 
\begin{exe}
  \ex a room \p{with} 2 beds [beds are among the things in the room]
  \ex\rf{Characteristic}{Stuff} where the object of the preposition is construed as describing the contents in their entirety:\begin{xlist}
    \ex a shelf \p{of} rare books
    \ex a cardboard box \p{of} snacks
  \end{xlist}
\end{exe}
\item Member(s) forming a partial subset of an organizational collective denoted by the governor:
\begin{exe}
  \ex	A piano quintet is a chamber group \p{with} a piano (in it)\\ (\rf{OrgRole}{Characteristic})
\end{exe}
\item With a transitive verb like \emph{search}, \emph{examine}, or \emph{test}, 
the attribute of the \psst{Theme} that is being examined:
\begin{exe}
  \ex He examined the vase \p{for} damage.
  \ex\label{ex:search-obj-for} He searched the room \p{for} his laser pistol. [contrast intransitive \psst{Theme}, \cref{ex:search-for}]
  \ex He was tested \p{for} low blood sugar.
\end{exe}
\item The scale or dimension by which items are compared:
\begin{exe}
  \ex The children are \choices{sorted\\screened} \p{by} height
  \ex\begin{xlist}
    \ex She exceeds him \p{in} height
    \ex There is no difference \p{in} height
  \end{xlist}
\end{exe}
\item	Anything that is borderline between the \psst{Possession} and \mbox{\psst{PartPortion}} subcategories
\end{itemize}

Typically, one of ``\psst{Gestalt} \{HAS, CONTAINS\} \psst{Characteristic}'' is entailed. 
This does not help to distinguish subtypes.

\begin{history}
  The v1 label \sst{Attribute} was intended to apply to features of something, 
  but was vaguely defined. With the overhaul of the \psst{Configuration} 
  subhierarchy, \sst{Attribute} has primarily been replaced by 
  \psst{Characteristic} and its subtypes and \psst{Identity}.
\end{history}

