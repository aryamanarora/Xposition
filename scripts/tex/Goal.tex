\section{Goal}

\shortdef{Final location (destination), condition, or value. May be abstract.}

Prototypical prepositions include \p{to}, \p{into}, and \p{onto}:
\begin{exe}
  \ex I ran \p{to} the store.
  \ex The cat jumped \p{onto} the ledge.
  %\ex a blow/bullet \p{to} the head. (\psst{Goal}? Theme ~> Goal for 'blow'?)
  \ex I touched my ear \p{to} the floor.
  \ex She sank \p{to} her knees.
  \ex Add vanilla extract \p{to} the mix.
  \ex Everyone contributed \p{to} the meeting.
  \ex The temperature is rising \p{to} a high of 40 degrees.
  \ex We have access \p{to} the library's extensive collections.
  \ex She slipped \p{into} a coma.
  \ex The drugs put her \p{in} a coma. (\rf{Goal}{Locus})
  \ex \textbf{Result} \citep[p.~1224]{cgel}: \begin{xlist}
    \ex We arrived at the airport only \p{to} discover that our flight had been canceled.
    \ex May you live \p{to} be 100!
  \end{xlist}
\end{exe}
For motion events, a \psst{Goal} must have been reached if the event 
has progressed to completion (was not interrupted).
\psst{Direction} is used instead for \p{toward(s)} and \p{for}, 
which mark an intended destination that is not necessarily reached:
\begin{exe}
  \ex\begin{xlist}
    \ex I headed \p{to} work. (\psst{Goal})
    \ex I headed \choices{\p{towards}\\\p{for}\\\#\p{to}} work but never made it there. (\psst{Direction})
  \end{xlist}
\end{exe}

% \paragraph{\emph{refer to} RESOURCE.}
% When \emph{refer to} means `mention' or `use a term for' \nss{TODO}
% When \emph{refer to} means `consult' or `advise to consult' a source of information such as a book,
% it is considered a multiword verbal expression (because the \p{to} cannot be omitted in adverbial questions\nss{Is it relevant that \emph{there} can be substituted?}), 
% so the \p{to} is not labeled with a supersense:
% \begin{exe}
%   \ex I referred\_to the dictionary. (\#Why did you refer?)
%   \ex I was referred\_to the dictionary. (\#Why were you referred?)
% \end{exe}
% When it means `make a referral', i.e.~`direct somebody to a business or service-provider', 
% and \p{to} marks the suggested business or service-provider, it is labeled \psst{Goal}.
% However, when \p{to} marks the recommendation-seeker, it is labeled \psst{Recipient}{Goal}:
% \begin{exe}
%   \ex\begin{xlist}
%     \ex I needed a hairdresser, so my friend referred me \p{to} \choices{Natasha\\Hair Inc.}. (\psst{Goal})
%     \ex I am a great hairdresser---please refer me \p{to} your friends! (\rf{Recipient}{Goal})
% \end{exe}

\paragraph{\emph{go to}.} A conventional way to express one's status as a student at some school is 
with the expression \pex{go \p{to} (name or kind of school)}.
Construal is used when \pex{go \p{to}} indicates student status, rather than 
(or in addition to) physical attendance:
\begin{exe}
  \ex\label{ex:student} I went \p{to} (school at$_{\psst{Locus}}$) UC Berkeley. (\rf{OrgRole}{Goal})
  \exp{ex:student} I went \p{to} UC Berkeley for the football game. (\psst{Goal})
\end{exe}
Going to a business as a customer, going to an attorney as a client, 
going to a doctor as a patient, etc.\ can also convey long-term status, 
but there is considerable gray area between habitual going and 
being in a professional relationship, so we simply use \psst{Goal}:
\begin{exe}
  \ex I go \p{to} Dr.~Smith for my allergies. (\psst{Goal})
\end{exe}

\paragraph{Locative as destination.}
English regularly allows canonically static locative prepositions to mark 
goals with motion verbs like \pex{put}.
We use the \rf{Goal}{Locus} construal to capture both the static and dynamic aspects of meaning:
\begin{exe}
  \ex\rf{Goal}{Locus}: \begin{xlist}
    \ex I put the lamp \p{next\_to} the chair.
    \ex I'll just hop \p{in} the shower.
    \ex I put my CV \p{on} the internet.
    \ex The cat jumped \p{on} my face.
    \ex The box fell \p{on} its side.
    \ex We arrived \p{at} the airport.
  \end{xlist}
\end{exe}

\paragraph{Application of a substance.}
\begin{xexe}
  \ex the paint that was applied \p{to} the wall (\psst{Goal})
  \ex the paint that was sprayed \p{onto} the wall (\psst{Goal})
  \ex the paint that was sprayed \p{on} the wall (\rf{Goal}{Locus}) % passive to avoid ambiguity with [the paint on the wall]NP
\end{xexe}
The wall is the endpoint of the paint, hence \psst{Goal} is the scene role. 
(Though the wall can be said to be affected by the action, we prioritize 
the motion aspect of the scene in choosing \psst{Goal} rather than \psst{Theme}.)

% Examined COCA first 100 results for "for London" and "for Paris".
% 'leave' is the dominant verb
%leave/flee/depart/embark/take off/set out/set sail/... for, board a plane for; head/make for; bound for; train/bus for

\psst{Goal} is prototypically inanimate, though it can be used to construe animate \psst{Participant}s 
(especially \psst{Recipient}).
Contrasts with \psst{Source}.

