\section{ComparisonRef}

\shortdef{The reference point in an explicit comparison (or contrast), i.e., 
an expression indicating that something is 
\textbf{similar/analogous to}, \textbf{different from}, or \textbf{the same as}
something else.}

The marker of the ``something else'' (the ground in the figure–ground relationship) 
is given the label \psst{ComparisonRef}:
\begin{exe}
  \ex \begin{xlist}
    \ex She is taller \p{than} me.
    \ex She is taller \p{than} I am.
    \ex She is taller \p{than} she is wide.
    \ex She is better at math \p{than} at drawing.
    \ex The shirt is more gray \p{than} black.
    %\ex She is greater in height \p{than} me.
  \end{xlist}
  \ex\label{ex:comparAs} \begin{xlist}
    \ex She is as tall \p{as} I am.
    \ex Your face is (as$_{\text{\rf{Characteristic}{Extent}}}$) red \p{as} a rose. (more on \p{as}-\p{as} comparatives: \cref{sec:as-as})
    \ex Your surname is the\_same \p{as} mine.
  \end{xlist}
  \ex Harry had never met anyone quite \p{like} Luna.
  \ex It was \choices{\p{as\_if}\\\p{like}} he had insulted my mother.
\end{exe}

The comparison is often made with respect to some dimension or attribute, the \psst{Characteristic}, 
which may or may not be scalar. 
The comparison may be figurative, employing simile, hyperbole, or spatial metaphor 
(\pex{close to} in the sense of `similar to'). 
The \psst{ComparisonRef} may even be a desirable or hypothetical/irrealis 
event or state (\pex{It was \p{as} it should have been}).

Prototypical prepositions include \p{than}, \p{as} (including the second item 
in the \p{as}--\p{as} construction), \p{like}, \p{unlike}. 
Prominent construals are \p{to} (\psst{Goal} for similar-thing) 
and \p{from} (\psst{Source} for dissimilar-thing).

\paragraph{\psst{Locus} construal.}
If something is preferred or appreciated \p{over} something else, \rf{ComparisonRef}{Locus} is used:
\begin{exe}
  \ex I prefer this restaurant \p{over} that one. (\rf{ComparisonRef}{Locus})\\{} 
  [paraphrase: I like this restaurant better \p{than} that one.]
\end{exe}
But for scenes of choice and substitution, see \psst{InsteadOf}.

\paragraph{\psst{Source} and \psst{Goal} construals.}
Resemblance and equivalence may be expressed with \p{to}, 
while difference may be expressed with \p{from}:
\begin{exe}
  \ex\rf{ComparisonRef}{Goal}:\begin{xlist}
    \ex Shall I compare thee \p{to} a summer's day?
    \ex Her height is \choices{equal\\close} \p{to} mine.
  \end{xlist}
  \ex\rf{ComparisonRef}{Source}:\begin{xlist}
    \ex We need to distinguish what is achievable \p{from} what is desirable.
    \ex Her height is different \p{from} mine.\footnote{American English. Interestingly, 
    \emph{different \p{to}} occurs in British English.}
  \end{xlist}
\end{exe}

\paragraph{\psst{Accompanier} construal.}
\begin{exe}
  \ex Don't compare me \p{with} my sister! (\rf{ComparisonRef}{Accompanier})
\end{exe}

\paragraph{Category as standard.} 
An indirect comparison can be made by relating something to a category 
to which it may or may not belong. 
The category stands for its members or prototypes. For example, in:
\begin{exe}
  \ex\label{ex:catAsStandard} He is short \p{for} a basketball player. (\psst{ComparisonRef})
\end{exe}
the category \pex{basketball player} serves as the standard against which \pex{he} is deemed short.

\paragraph{Sufficiency and excess.}
In a statement of sufficiency or excess, 
\rf{ComparisonRef}{Purpose} marks the consequence enabled or prevented by some condition.
This is similar to \textbf{necessity} (discussed under \psst{Purpose}), 
only here, the condition involves a comparison; the standard of comparison 
is implied by the consequence:
\begin{exe}
  \ex\rf{ComparisonRef}{Purpose}:\begin{xlist}
    \ex\label{ex:tooShort} He is \choices{too short\\not tall enough} \choices{\p{for}\\\p{to} play} basketball.
    \ex\label{ex:insufficient} His height is insufficient \p{for} basketball.
  \end{xlist}
\end{exe}
In these constructions, an adverb (\pex{too}, \pex{enough}, \pex{insufficiently}, etc.)\ 
or an adjective (\pex{insufficient}) licenses the PP or infinitival expressing the consequence.\footnote{See the Degree-Consequence construction \citep{bonial-18}.}
They amount to saying
\pex{He is shorter than the height he would have to be in order to play basketball}, 
which features separate constructions for comparison and necessity-for-purpose.

\paragraph{\rf{Manner}{ComparisonRef} construal.}
This applies to an analogy that describes the \emph{how} of an event 
(be it agentive or perceptual):
\begin{exe}
\ex \rf{Manner}{ComparisonRef}:\begin{xlist}
  \ex You eat \p{like} a pig (eats).
  \ex You smell \p{like} a pig.
\end{xlist}
\end{exe}
However, where an analogy is an external comment on an event 
rather than filling in a role of the event, it is simply \psst{ComparisonRef}. 
Contrast:
\begin{exe}
  \ex You ate a whole pie \p{like} my cousin did.
  \begin{xlist}
    \ex \emph{Role reading:} The way in which you ate a pie was similar. (\rf{Manner}{ComparisonRef})
    \ex \emph{External comment reading:} You ate a whole pie, and so did my cousin. (\psst{ComparisonRef})
  \end{xlist}
  %\ex I was elated, \p{like}/\p{as\_if} I had just won the lottery. (\psst{ComparisonRef})
\end{exe}

\paragraph{Analogy and non-analogy readings of \p{like}.}
In descriptions, adverbial \p{like}, \p{as\_if}, etc.\  %with an extracted object of \pex{what} 
can be ambiguous, especially in a scene of perception. 
For example:
\begin{exe}
  \ex This looks \p{like} a Van Gogh painting.
  \begin{xlist}
    \ex \emph{Analogy reading:} This looks similar to a Van Gogh painting. (\rf{Manner}{ComparisonRef})
    \ex \emph{Conclusion reading:} This looks to be a Van Gogh painting (it probably is one). (\rf{Theme}{ComparisonRef})
  \end{xlist}
  \ex It sounded \p{like}/\p{as\_if}
  \begin{xlist}
    \ex \dots he had drunk a gallon of helium. (\rf{Manner}{ComparisonRef}: analogy reading more likely)
    \ex \dots they weren't taking me seriously. (\rf{Theme}{ComparisonRef}: conclusion reading more likely)
  \end{xlist}
\end{exe}
Similarly for \pex{seem \p{like}}, \pex{feel \p{like}}, etc.

Another ambiguity can arise when \p{like} occurs with \pex{what} as its extracted object. 
In the following sentences, the most likely interpretation is not one of analogy between 
two things, but rather an open-ended description. 
(\pex{Who does it look \p{like}?}, by contrast, implicates an analogy to an individual.) 
We therefore treat \pex{\p{like} what} as a PP idiom, 
and label it \rf{Manner}{ComparisonRef}:
\begin{exe}
  \ex\label{ex:whatlike}\rf{Manner}{ComparisonRef}:\begin{xlist}
    \ex I know what\_~~Steve looks~~\_\p{like}. (I know how Steve looks.)
    \ex What\_~~does her hair look~~\_\p{like}? (How does her hair look?)
    \ex What\_~~is the party~~\_\p{like}? (How is the party?)
  \end{xlist}
\end{exe}
A \pex{how}-paraphrase is generally possible, though \pex{how} may suggest 
a positive or negative evaluation is available, whereas \pex{what} is more neutral.

Constrast unaccusative perception verb + \p{of} combinations:
\begin{exe}
  \ex\label{ex:smellOfNotCmp} \choices{Your father smells\\The soup tastes} \p{of} elderberries. (\rf{Manner}{Stuff}) [also~\cref{ex:smellOf}]
\end{exe}

\paragraph{Category exemplars and set members.} 
When governed by an NP naming a category or set, \p{like} is ambiguous 
between exemplifying a member, as in \cref{ex:likeSetMember} and \cref{ex:likeCatMember}, 
and merely indicating similarity, as in \cref{ex:likeSetSimilar} and \cref{ex:likeCatSimilar}:
\begin{exe}
  \ex Colbert frequently promotes comedians \p{like} himself.
    \begin{xlist}
      \ex\label{ex:likeSetSimilar} [\emph{Exclusive/restrictive reading:} \emph{similar to} himself (but not including himself)]
        (\psst{ComparisonRef})
      \ex\label{ex:likeSetMember} [\emph{Inclusive/nonrestrictive reading:} \emph{such as}/\emph{including} himself (he promotes himself, among others)]
        (\rf{PartPortion}{ComparisonRef})
    \end{xlist}
  \ex 
    \begin{xlist}
      \ex\label{ex:likeCatSimilar} I don't know anyone else \p{like} her. [anyone else \emph{similar to} her]\\ (\psst{ComparisonRef})
      \ex\label{ex:likeCatMember} It must be great to have a wonderful doctor \p{like} \choices{her\\she is}.\\ {}
      [It must be great to have her because she is a wonderful doctor]\\ (\rf{Identity}{ComparisonRef})
    \end{xlist}
\end{exe}

