\section{Co-Theme}

\shortdef{Second semantically core undergoer that would otherwise be labeled \psst{Theme}, 
but which is adpositionally marked in contrast with a \psst{Theme} 
occupying a non-oblique syntactic position (subject or object).}

Often, the \psst{Theme} and the \psst{Co-Theme} are similarly situated entities---rather than 
one being more figure-like and the other more ground-like---but the \psst{Co-Theme} 
is an oblique (adpositionally marked) argument.
This includes concrete scenes of combination, attachment, separation, and substitution 
of two similar entities. 

\begin{exe}
  \ex\begin{xlist}
    \ex His bicycle collided \p{with} hers.
    \ex Combine butter \p{with} vanilla.
    \ex\label{ex:repl-with} They replaced my old tires \p{with} new ones. [replacement; contrast \cref{ex:subst-for}]
  \end{xlist}
  \ex\begin{xlist}
    %\ex You can't create something \p{from} nothing. (\rf{Co-Theme}{Source})
    \ex The boys were separated \p{from} the girls. (\rf{Co-Theme}{Source})
    \ex Keep the dogs \p{from} the cats. (\rf{Co-Theme}{Source})
    \ex The shin bone is connected \p{to} the knee bone. (\rf{Co-Theme}{Goal})
  \end{xlist}
\end{exe}

By contrast, for similar scenes where the oblique argument is a ground-like entity 
(larger, less dynamic, more locational, etc.\ than the \psst{Theme}), 
that entity is typically a \psst{Locus}, \psst{Source}, or \psst{Goal}:
\begin{exe}
  \ex Dynamic:\begin{xlist}
    \ex Add vanilla \p{to} the mixture. (\psst{Goal})
    \ex Stir vanilla \p{into} the mixture. (\psst{Goal})
    \ex Detach the cable \p{from} the wall. (\psst{Source})
  \end{xlist}
  \ex Static:\begin{xlist}
    \ex The cable \choices{is attached\\connects} \p{to} the wall. (\rf{Locus}{Goal})
    \ex Protesters were \choices{kept\\missing} \p{from} the area. (\rf{Locus}{Source}) [repeated: \cref{ex:protesters}]
  \end{xlist}
\end{exe}

For creation or transformation of a whole entity (or a group of entities, such as ingredients) 
into another entity, \psst{Source} applies to the initial entity and \psst{Goal} to the result.

With abstract scenes, \psst{Co-Theme} is sometimes needed because another 
argument would be \psst{Theme}---e.g.~2-argument adjectives:
\begin{exe}
  \ex\begin{xlist}
    \ex You shouldn't confuse/associate Mozart \p{with} Rossini. (\psst{Co-Theme})
    \ex We are ready/eligible/due \p{for} an upgrade. (\rf{Co-Theme}{Purpose})
    \ex They prevented us \p{from} entering. (\rf{Co-Theme}{Source})
  \end{xlist}
\end{exe}

\begin{history}
  In v1, \sst{Co-Patient} was a distinct label, and the two shared a common supertype, 
  \sst{Co-Participant}. 
  See note at \psst{Theme}.
\end{history}

See also: \psst{InsteadOf}, \psst{Co-Agent}

