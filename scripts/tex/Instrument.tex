\section{Instrument}

\shortdef{An entity that facilitates an action by applying intermediate causal force.}

Prototypically, an \psst{Agent} intentionally applies the \psst{Instrument} 
with the purpose of achieving a result:
\begin{exe}\ex\begin{xlist}
  \ex I broke the window \p{with} a hammer.
  \ex I destroyed the argument \p{with} my words.
\end{xlist}\end{exe}
Less prototypically, the action could be unintentional:
\begin{exe}
  \ex I accidentally poked myself in the eye \p{with} a stick.
\end{exe}
The key is that the \psst{Instrument} is not sufficiently ``independently causal'' 
to instigate the event.

However, to downplay the agency of the individual operating the instrument, 
the instrument can be placed in a passive \p{by}-phrase, 
which construes it as the instigator:
\begin{exe}\ex\label{ex:passiveInstrument}\begin{xlist}
  \ex The window was broken \p{by} the hammer. (\rf{Instrument}{Causer})
  \ex My headache was alleviated \p{by} aspirin. (\rf{Instrument}{Causer})
\end{xlist}\end{exe}
Note that the examples in \cref{ex:passiveInstrument} can be rephrased 
in active voice with the \psst{Instrument} as the subject.

A device serving as a mode of transportation or medium of communication 
counts as an \psst{Instrument}, but is often construed as a \psst{Locus} or \psst{Path}:
\begin{exe}
  \ex Communicate \p{by} \choices{phone\\email}. (\psst{Instrument})
  \ex Talk \p{on} the phone. (\rf{Instrument}{Locus})
  \ex Send it \choices{\p{over}\\\p{via}} email. (\rf{Instrument}{Path})
  \ex Travel \p{by} train. (\psst{Instrument})
  \ex Escape \p{with} a getaway car. (\psst{Instrument})
  \ex Escape \p{in} the getaway car. (\rf{Instrument}{Locus})
\end{exe}
This includes some expressions which incorporate the \psst{Instrument} 
in a noun:
\begin{exe}
  \ex ride \p{on} horseback (\rf{Instrument}{Locus})
  \ex hold \p{at} knifepoint (\rf{Instrument}{Locus})
\end{exe}
Other non-prototypical instruments that can be construed as paths 
include waypoints from \psst{Source} to \psst{Goal}, 
and people\slash organizations serving as intermediaries:
\begin{exe}
  \ex We flew to London \p{via} Paris. (\rf{Instrument}{Path})
  \ex I found out the news \p{via} Sharon. (\rf{Instrument}{Path})
  \ex Joan bought her house \p{through} a real estate agent. (\rf{SocialRel}{Instrument})
  \ex For my Honda I always got replacement parts \p{through} the dealership. (\rf{OrgRole}{Instrument})
\end{exe}

Conversely, roadways count as \psst{Path}s but can be construed as \psst{Instrument}s:
\begin{exe}
  \ex Escape \p{through} the tunnel. (\psst{Path})
  \ex Escape \p{by} tunnel. (\rf{Path}{Instrument})
\end{exe}

Compare \psst{Means}, which is used for facilitative events rather than entities.
See also \psst{Topic}.

