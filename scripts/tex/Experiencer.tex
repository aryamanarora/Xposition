\section{Experiencer}

\shortdef{Animate who is aware of a bodily experience, perception, emotion, or mental state.}

\psst{Experiencer} does not seem to have any prototypical adposition 
in the languages we have looked at. In English, it can be construed in several ways:
\begin{exe}
  \ex\begin{xlist}
    \ex The anger \p{of} the students (\rf{Experiencer}{Gestalt})
    \ex The student\p{s'} anger (\rf{Experiencer}{Gestalt})
  \end{xlist}
  \ex\begin{xlist} 
    \ex Running is enjoyable \p{for} me (\rf{Experiencer}{Beneficiary})
    \ex The pizza was (too) salty \p{for} me (\rf{Experiencer}{Beneficiary})
  \end{xlist}
  \ex\begin{xlist}
    \ex It feels hot \p{to} me (\rf{Experiencer}{Goal})
    \ex That was astounding \p{to} me (\rf{Experiencer}{Goal})
  \end{xlist}
\end{exe}

Less canonically, \psst{Experiencer} applies to semi-pragmatic usages meaning `from the perspective of':\footnote{Interestingly, 
many uses of \p{for} carry an information structural association of delimiting the scope of an assertion. 
\pex{\p*{For}{for} John, the party was not fun at all} makes no commitment regarding how fun the party was to others. 
\pex{This food is good \p{for}$_{\psst{Purpose}}$ dinner\slash \p{for}$_{\psst{Beneficiary}}$ folks with dietary constraints} 
and \pex{He is short \p{for}$_{\psst{ComparisonRef}}$ a basketball player} also have this property. 
As the present scheme targets semantic relations, it is not equipped to formalize pragmatic aspects of the meaning.}
\begin{exe}
  \ex \begin{xlist}
    \ex \p*{For}{for} John, the party was not fun at all. (\rf{Experiencer}{Beneficiary})
    \ex \p*{For}{for} John, there was no reason to attend. (\rf{Experiencer}{Beneficiary})
  \end{xlist}
\end{exe}

Elsewhere, the term \emph{cognizer} is sometimes used for one whose 
mental state is described.

Counterpart: \psst{Stimulus}

