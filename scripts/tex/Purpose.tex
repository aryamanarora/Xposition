\section{Purpose}

\shortdef{Something that somebody wants to bring about,
asserted to be why something was done, is the case, or exists.}

Central usages of \psst{Purpose} explain the motivation behind an action.
Typically the governing event serves as a means for achieving or facilitating the \psst{Purpose}. 
Prototypical markers include \p{for} and infinitive marker \p{to}:
\begin{exe}
  \ex\begin{xlist}
      \ex He rose \p{to} make a grand speech.
      \ex He rose \p{for} a grand speech.
      \ex surgery \p{to} treat a leg injury
    \end{xlist}
\end{exe}
Something directly manipulated/affected can stand in metonymically 
for the desired event:
\begin{exe}\ex\begin{xlist}
  \ex I went to the store \p{for} eggs. [understood: `to acquire/buy eggs']
  \ex surgery \p{for} a leg injury [understood: `to treat a leg injury']
\end{xlist}\end{exe}

\noindent The following subcases serve to clarify the boundaries of \psst{Purpose}:
\begin{itemize}
\item	A desired outcome that is separate from, but typically a motivation for 
(hence subtype of \psst{Explanation}), the main event. 
It is possible to complete the main event without realizing the purpose.
\item \textbf{Inanimate} thing or event which is aided\slash facilitated\slash addressed\slash achieved\slash acquired 
as a consequence of the main event:
\begin{exe}
  \ex We hired a caterer \p{for} the party.
  \ex surgery \p{for} an ingrown toenail
  \ex Call the doctor \p{for} an appointment.
  \ex Go to the store \p{for} eggs.
\end{exe}
For \textbf{animates} who are aided or harmed as a consequence of the main event, see \psst{Beneficiary}.
\item	Something characterized as good/appropriate (or not) for some kind of use\slash activity\slash occasion 
or inanimate thing affected\slash addressed\slash etc., 
delimiting the applicability of a descriptor to that aspect of the thing:
\begin{exe}
  \ex \begin{xlist}
      \ex This place is great \p{for} ping-pong. 
      %[no commitment to evaluating the place in general]
      \ex This is a great place \p{for} ping-pong.
    \end{xlist}
  \ex This cleaner is good \p{for} hardwood floors.
  \ex\label{ex:greatForDinner} This restaurant is great \p{for} dinner.
\end{exe}
The evaluation is being delimited to a particular purpose:
\cref{ex:greatForDinner} is not claiming the restaurant is great \emph{in general}, 
just with respect to dinner.

For \textbf{animates} used in similar constructions, see \psst{Beneficiary}.
\end{itemize}

%●	Intended use of something. “a shoulder TO cry on”, “a shoulder FOR crying on": Characteristic ~> Purpose

Question test: \psst{Explanation} and its subtype \psst{Purpose}, 
when used adverbially, license \pex{Why?} questions. 
\psst{Purpose} usually licenses an \pex{in order to} 
or \pex{for the purpose of} paraphrase.

\paragraph{Goods and services.} See discussion at \psst{Theme}.

\paragraph{Inherent purposes.}
An \emph{entity} (typically an artifact) can be modified to explicate an intended use or affordance. 
We analyze such a relationship with the construal \rf{Characteristic}{Purpose}: 
the function of \psst{Purpose} reflects the \emph{intended use} aspect of the meaning, 
while the scene role of \psst{Characteristic}
highlights that the intended use can be understood as a \emph{static property} of the entity (part of its qualia structure).%
\footnote{In FrameNet 
as of v1.7, these sorts of purposes are labeled as \sst{Inherent\_purpose}. 
See, e.g., the example ``MONEY [to support yourself and your family]'' in the \textbf{Money} frame 
(\url{https://framenet2.icsi.berkeley.edu/fnReports/data/lu/lu13361.xml?mode=annotation}).}
\begin{exe}
  \ex\label{ex:charPurp} \rf{Characteristic}{Purpose}:
  \begin{xlist}
    \ex a shoulder \p{to} cry on
    \ex something \p{to} eat
    \ex cleaner \p{for} hardwood floors
%    \ex a good store \p{for} eggs [understood: `for acquiring/buying eggs']
%    \ex a good book \p{to} give to young readers
%    \ex a good book \p{for} young readers [understood: `for giving to young readers']
    %\ex physical therapy \p{for} a leg injury. [understood: `treating a leg injury']
  \end{xlist}
\end{exe}

Question test: \pex{What is this \_ for?}


\paragraph{Necessity.}
\psst{Purpose} marks the consequence enabled or prevented by some condition, 
typically a resource:
\begin{exe}
  \ex I need another course \p{in\_order\_to} graduate. (\psst{Purpose})
  \ex I need \$20 \choices{\p{for}\\\p{to} attend} the show. (\psst{Purpose})
  \ex The dough takes an hour \p{to} rise. (\psst{Purpose})
\end{exe}

\paragraph{Sufficiency and excess.}
See \psst{ComparisonRef}.


\paragraph{Versus \psst{Circumstance} for ritualized occasions.}
\psst{Purpose} applies to \p{for} 
when it marks a ritualized activity such as a meal or holiday/commemoration for which the main event describes a \textbf{preparation} stage:
%ORIGINAL HERE: provided that the main event is in preparation for the observance: 
\begin{exe}
  \ex \psst{Purpose}:
    \begin{xlist}
      \ex I walked to this restaurant \p{for} dinner. [walking is not a part of dinner]
      \ex I bought food \p{for} dinner.
      \ex We saved money \p{for} our annual vacation.
    \end{xlist}
\end{exe}
However, if the activity marked by \p{for} is interpreted as \textbf{containing} the main event, 
then we use \psst{Circumstance}: 
\begin{exe}
  \ex \psst{Circumstance}: 
    \begin{xlist}
      \ex We ate there \p{for} dinner.
      \ex I received a new bicycle \p{for} Christmas.
      \ex I always drink eggnog \p{for} Christmas. [at and in celebration of Christmastime]
      \ex We were wearing costumes \p{for} Halloween. 
    \end{xlist}
\end{exe}
If in doubt, \psst{Circumstance} is broader: e.g., \emph{We went there \p{for} dinner} 
if \emph{went} is ambiguous between journeying and attending.


\begin{history}
  In v1, the usages illustrated in \cref{ex:charPurp} were assigned a separate label, 
  \sst{Function}, which inherited from both \sst{Attribute} and \psst{Purpose}.
  The ability to use construal removes the need for a separate label.
\end{history}

