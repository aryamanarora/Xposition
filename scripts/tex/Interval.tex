\section{Interval}

\shortdef{A marker that points retrospectively or prospectively in time, 
and if transitive, marks the time elapsed between two points in time.}

The clearest example is \p{ago}, which only serves to locate the \psst{Time}
of some past event in terms of its distance from the present:
\begin{exe}
  \ex\label{ex:ago} I arrived a year \p{ago}. (\rf{Time}{Interval}) \\{}
  [points backwards from the present: before now]
\end{exe}
The most common use of \psst{Interval} is in the construal \rf{Time}{Interval}: 
the time of an event is described via a temporal offset from some other time.

Another retrospective marker, \p{back}, can be transitive \cref{ex:backTrans}, 
or can be an intransitive modifier 
of a \psst{Time} PP \cref{ex:backIntrans}. 
Plain \psst{Interval} is used in the latter case:
\begin{exe}
  \ex\label{ex:backTrans} I arrived a year \p{back}.\footnote{While 
  \pex{a while \p{back}} and \pex{a few generations \p{back}} are generally accepted, 
  %\p{back} with smaller measurements,
  the use of \p{back} rather than \p{ago} for nearer and more precise temporal references,
  e.g.~\pex{10~minutes \p{back}}, appears to be especially associated with Indian English \citep[p.~7]{yadurajan-01}.} (\rf{Time}{Interval})
  \ex\label{ex:backIntrans} I arrived \p{back} in$_{\psst{Time}}$ June. (\psst{Interval})
\end{exe}



(This category is unusual in primarily marking a construal for a different scene role. 
But this seems justified given the restrictive set of English temporal prepositions 
that can appear with a temporal offset, and the distinct ambiguity of \p{in}.
\psst{Interval} is designed as the temporal counterpart of \psst{Direction}, 
which can construe static distance measures; 
in fact, \sst{TimeDirection} was considered as a possible name, 
but \psst{Interval} seemed more straightforward for the most frequent class of usages.)

Other adpositions can also take an amount of intervening time as their \emph{complement} (object):
\begin{exe}
  \ex\label{ex:inAmbiguousTime} I will eat \p{in} 10~minutes.
    \begin{xlist}
      \ex\label{ex:inDuration} {} [`for no more than 10~minutes' reading]: \psst{Duration}\footnote{This usage of \p{in} has been classified under the terms \emph{frame adverbial} \citep{pustejovsky-91} and \emph{span adverbial} \citep{chang-98}.}
      \ex\label{ex:inInterval} {} [`10~minutes from now' reading]: \rf{Time}{Interval}\footnote{This usage of \p{in}, as well as \p{ago} \cref{ex:ago} and \p{back} \cref{ex:backTrans,ex:backIntrans},
      are \emph{deictic}, i.e., they are inherently relative to the speech time or deictic center. 
      (See also \citet[pp.~154--157]{klein-94}.)
      This was taken to be a criterion for the v1 category \sst{DeicticTime}, 
      but that was never well-defined in v1 and was broadened for this version.}
    \end{xlist}
  % \ex\begin{xlist}\nss{not sure what to do with these. remove for now}
  %   \ex What are the revenue projections 6~months \p{out}?
  %   \ex I've started watching a new TV series and am 3~episodes \p{in}.\nss{``3 episodes into the show''?}
  %\end{xlist}
  \ex\label{ex:AfterObj} The game started at 7:00, but I arrived \choices{\p{after}\\\p{within}} 20~minutes. (\rf{Time}{Interval})
\end{exe}
Some adpositions license a temporal difference measure in \emph{modifier} position, which does not qualify:
\begin{exe}
  \ex To beat the crowds, I will arrive \uline{a while} \choices{\p{before} (it starts)\\\p{beforehand}}. (\psst{Time})
  \ex\label{ex:AfterMod} The game started at 7:00, but I arrived \uline{20~minutes} \choices{\p{after} (it started)\\\p{afterward}}. (\psst{Time})
\end{exe}
The preposition \p{after} can be used either way---contrast \cref{ex:AfterMod} with \cref{ex:AfterObj}.

Note that having \psst{Interval} as a separate category allows us to distinguish the sense of \p{in} 
in \cref{ex:inInterval} from both the \psst{Duration} sense \cref{ex:inDuration} 
and the \psst{Time} sense (\pex{\p{in} the morning}).

\paragraph{Versus \psst{Duration}.} 
The prepositions \p{in} and \p{within} are ambiguous between \psst{Interval} and \psst{Duration}.\footnote{By contrast, 
\p{after} seems to strongly favor \rf{Time}{Interval}. 
\pex{\p*{After}{after} a week, I had climbed all the way to the summit} is possible, 
but the conclusion that the climbing took a week may be an inference 
rather than something that is directly expressed.}
The distinction can be subtle and context-dependent.
The key test is whether the phrase answers a \emph{When?}\ question. 
If so, its scene role is \psst{Time}; otherwise, it is a \psst{Duration}.
%With a verb which refers to a punctual moment of culmination, 
%we use \psst{Interval} even though there may be an implicit 
%preparatory process with a duration:
\begin{exe}
  \ex \rf{Time}{Interval}:
  \begin{xlist}
    \ex I reached the summit \p{in} 3~days. [= 3~days later, I reached the summit.]
    \ex I was at the summit \p{within} 3~days. [= 3~days later, I was at the summit.]
    \ex I finished climbing \p{in} 3~days. [= 3~days later, I finished climbing.]
    \ex They had the engine fixed \p{in} 3~days. [= 3~days later, they had the engine fixed.]
  \end{xlist}
\end{exe}
\begin{exe}
  \ex \psst{Duration}:
    \begin{xlist}
      \ex I reached the summit \p{in} 3~days. [it took not more than 3~days]
      \ex I had climbed 1000 feet \p{in} [a total of] 3~days.
      \ex I fixed the engine \p{in} 3~days. [it took not more than 3~days] %\\
      %$\rightarrow$ I was fixing the engine \p{for} 3~days. (\psst{Duration})
    \end{xlist}
\end{exe}

%Though \p{for} generally marks \psst{Duration}s, it can mark an \psst{Interval} under negation:
With a negated event, we use \psst{Duration}:
\begin{exe}
  \ex I haven't eaten \choices{\p{in}\\\p{for}} hours. [hours have passed since the last time I ate]
  (\choices{\#When} haven't you eaten?)
\end{exe}

% Some temporal prepositions may be intransitive \cref{ex:IntrRefTime}, 
% or may take a temporal difference measure as the object \cref{ex:TimeDiff},
% provided that a reference time is salient in the discourse. 
% With such prepositions, the reference time may default to the speech time. 
% However, we limit \psst{DeicticTime} to those prepositions 
% whose \emph{inherent} reference point is the speech time. 
% The following contexts allow a \psst{Time} but not a \psst{DeicticTime}: 
% \begin{exe}
%   %\ex\label{ex:IntrRefTime} To beat the crowds at the game, I will arrive an hour \choices{\p{before}\\\p{beforehand}\\ \#\p{ago}}. (\psst{Time})
%   \ex\label{ex:IntrRefTime} To beat the crowds at the game, I will arrive a while \choices{\p{before}\\\p{beforehand}\\\#\p{ago}\\\#\p{back}}. (\psst{Time})
%   \ex\label{ex:TimeDiff} I took a seat in the waiting room. \choices{\p{After}\\\p{Within}\\??\p{In}} 15 minutes, the doctor saw me for an hour. (\psst{Time})
% \end{exe}
% To summarize: \psst{DeicticTime} covers preposition usages where `now' is inherent,
% and the more general \psst{Time} applies if the reference time is given in the 
% direct object or depends on the discourse.

\begin{history}
  Version~1 featured a label called \sst{DeicticTime}, under \sst{RelativeTime}, 
  which was meant to cover \p{ago} and temporal usages of other adpositions 
  (such as \p{in}) whose reference point is the utterance time or deictic center. 
  This concept proved difficult to apply and was (without good justification) 
  used as a catch-all for intransitive usages of temporal prepositions. 
  For v2, the new concept of \psst{Interval} is broader in that it drops the deictic 
  requirement (also covering \p{within}), while \psst{Time} has been clarified to include 
  intransitive usages of prepositions like \p{before} where the reference time 
  can be recovered from discourse context.
\end{history}


