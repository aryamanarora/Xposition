\section{Gestalt}

\shortdef{Generalized notion of ``whole'' understood with reference to 
a component part, possession, set member, or characteristic. 
See \psst{Characteristic}.}

\psst{Gestalt}---the supercategory of \psst{Whole} and \psst{Possessor}---applies 
directly for entities and eventualities which can loosely be conceptualized as 
containing or possessing something else, but for which 
neither \psst{Whole} nor \psst{Possessor} is a good fit.

\paragraph{Properties.}
The holder of a property if the property is the governor:
\begin{exe}
    \ex {\setlength\multicolsep{0pt}%
    \begin{multicols}{2}
      \begin{xlist} 
        \ex the blueness \p{of} the sky
        \ex the size \p{of} the crowd
        \ex the price \p{of} the tea
        \ex the start time \p{of} the party
        
        \sn the sky\p{'s} blueness
        \sn the crowd\p{'s} size
        \sn the tea\p{'s} price
        \sn the party\p{'s} start time
      \end{xlist}
    \end{multicols}}
    \ex\label{ex:amountGestalt} the amount \p{of} time allowed [but see \cref{ex:QuantityGestalt}]
    \ex the food/service \p{at} this restaurant (\rf{Gestalt}{Locus})
\end{exe}

\paragraph{Containers.}
The construal \rf{Locus}{Gestalt} is used for a container denoted by the governor:
\begin{exe}
\ex the room\p{'s} 2 beds (\rf{Locus}{Gestalt})
\end{exe}

\paragraph{Discourse-associated item.}
A referent temporarily associated with another referent in the discourse 
and used to help identify it: 
\begin{exe}
  \ex Sam\p{'s} dog (= the dog that Sam mentioned seeing earlier in the conversation)
\end{exe}
%\item	Anything that is borderline between subcategories \psst{Possessor} and \psst{Whole}

\paragraph{Other possessive constructions.}
\psst{Gestalt} is the construal for many uses of possessive syntax
where the semantic criteria for \psst{Possessor} are not met. 
For instance, s-genitive marking of participant roles (\psst{Agent}, \psst{Experiencer}, 
etc.)\ are analyzed with \psst{Gestalt} as the function. 
Moreover, the s-genitive construction, 
unlike \p{of}, is never analyzed with \psst{Whole} as the function, 
so \rf{Whole}{Gestalt} is used. 
See \cref{sec:genitives} for discussion of possessive constructions.

