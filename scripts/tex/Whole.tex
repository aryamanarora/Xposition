\section{Whole}

\shortdef{Something described with respect to its part, portion, subevent, subset, 
or set element. See \psst{PartPortion}.}

\begin{exe}
  
  \ex {\setlength\multicolsep{0pt}%
  \begin{multicols}{2} 
   \begin{xlist}
    % of-genitive
    \sn \psst{Whole}
    \ex	the new engine \p{of} the car
    \ex	the flaxen hair \p{of} the girl
    \ex\label{ex:layers}	the 3 layers \p{of} the cake
    \ex\label{ex:prongs}	the 3 prongs \p{of} the strategy
    \ex the tastiest bit \p{of} the cake
    \ex the southern tip \p{of} the island
    \ex the interior \p{of} the shopping bag
    \ex the end \p{of} the journey
    \ex the 14~episodes \p{of} a TV series
    
    % s-genitive
    \sn \rf{Whole}{Gestalt}
    \sn the car\p{'s} new engine
    \sn the girl\p{'s} flaxen hair
    \sn the cake\p{'s} 3 layers
    \sn the strategy\p{'s} 3 prongs
    \sn the cake\p{'s} tastiest bit
    \sn the island\p{'s} southern tip
    \sn the shopping bag\p{'s} interior
    \sn the journey\p{'s} end
    \sn a TV series\p{'s} 14~episodes
  \end{xlist}
  \end{multicols}}
  \ex	the south \p{of} France
    %\ex 2 \p{of} my 5 daughters % Quantity ~> Whole
  \ex\label{ex:rest} The \choices{remainder\\rest} \p{of} the cake %\nss{Maybe this should be 
    % \rf{Quantity}{Whole} after all, even though ``the rest of it'' is a dubious way to answer 
    % ``How much of it?''. ``The remaining 6 ounces of cake'' certainly specifies a quantity.\ab{but is that because of "6 ounces" or "remainder (\/remaining)"?   Since "remaining of the cake" is cake too (a physical object in this case), it has the property of having quantity, but "remaining of ..." does not have "quantity" as the sense in the foreground, it's part-whole relationship.}}
    %\ex	the tennis matches \p{of} a series
    %\ex	the beginning \p{of} the party
  \ex \rf{Whole}{Locus}: \begin{xlist}
    \ex	the 14~episodes \p{in} a TV series
    \ex	the new engine \p{in} the car
    \ex the escape key \p{on} the keyboard
    \ex the flaxen hair \p{on} the girl
  \end{xlist}
  \ex\label{ex:in-pile-adnominal}	the clothes \p{in} that pile are dirty (\rf{Whole}{Locus}) [but contrast adverbial/predicative \p{in}+shape: \cref{ex:in-shape} under \psst{Manner}]
  \ex There are several options to choose \p{from}. (\rf{Whole}{Source})
  \ex\label{ex:sets} Sets and ratios:
    \begin{xlist}
      \ex This is one \p{of} the \choices{worst\\better} retaurants in town. (\psst{Whole})
      \ex 2 \p{in} 10 American children are redheads. (\rf{Whole}{Locus})
      \ex 2 \p{out\_of} 10 American children are redheads. (\rf{Whole}{Source}) %\nss{should this be \psst{RateUnit}?}\ab{maybe}
      \ex \p*{Out\_of}{out\_of} the 10 children in the class, only Mary is a redhead. (\rf{Whole}{Source})
      \ex\label{ex:amongSet} \p*{Among}{among} the 10 children in the class, only Mary is a redhead. (\psst{Whole})
    \end{xlist}
\end{exe}

If the governor narrows the reference to a certain amount of the \psst{Whole}, 
the construal \rf{Quantity}{Whole} is used---see \cref{ex:QuantityWhole}. 
Note that this only applies if the governor is a measure term; 
it does not apply to distinctive parts like ``layers'' \cref{ex:layers} 
and ``prongs'' \cref{ex:prongs}, even if a count is specified.

Used to construe geographic and temporal ``containers'':
\begin{exe}
  \ex	Famous castles \p{of} the valley (\rf{Locus}{Whole})
  \ex \begin{xlist}
    \ex the \choices{15th\\Ides} \p{of} March (\rf{Time}{Whole})
    \ex March \p{of} 44~BC (\rf{Time}{Whole})
  \end{xlist}
\end{exe}

The prepositions \p{between} and \p{among} can impose \psst{Whole} construals 
by combining two or more items in the object NP (contrast with \cref{ex:amongSet}):
\begin{exe}
  \ex\label{ex:betweenParties}  The negotiations \choices{\p{between}\\\p{among}} the parties went well. (\rf{Agent}{Whole})
  \exp{ex:betweenParties} The negotiations \p{by} the parties went well. (\psst{Agent})
\end{exe}

\begin{history}
  In v1, \sst{Superset} was distinguished as a subtype of \psst{Whole} 
  for examples such as \cref{ex:sets}, but the distinction was dropped for v2 
  (as was \sst{Elements}: see \psst{PartPortion}).
\end{history}

