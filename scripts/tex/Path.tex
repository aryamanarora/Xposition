\section{Path}

\shortdef{The ground that must be covered in order for the motion to be complete.}

The ground covered is often a linear extent with or without 
specific starting and ending points:
\begin{exe}
  \ex The bird flew \p{over} the building.
  \ex The sun traveled \p{across} the sky.
  \ex Hot water is running \p{through} the pipes.
\end{exe}

It can also be a waypoint\slash something that must be passed or encircled. 
\begin{exe}
  \ex We flew to Rome \p{via} Paris.
  \ex I go \p{by} that coffee shop every morning.
  \ex The earth has completed another orbit \p{around} the sun.
\end{exe}
If this is a portal in the boundary of a container, 
it is often construed as \psst{Source}, \psst{Goal}, or \psst{Locus}:
\begin{exe}
  \ex The bird flew \p{in} the window. (\rf{Path}{Locus})
  \ex The bird flew \p{out} the window. (\rf{Path}{Source})
  \ex A cool breeze blew \p{into} the window. (\rf{Path}{Goal})
\end{exe}

The prepositions \p{around} and \p{throughout} can mark a region in which motion 
that follows an aimless or complex trajectory is contained. 
%and which it roughly ``covers'' via an . 
Construal is used for these, whether or not the region is explicit:
\begin{exe}\ex \rf{Locus}{Path}:\begin{xlist}
  \ex The kids ran \p{around}.
  \ex The kids ran \choices{\p{around}\\\p{throughout}} the kitchen.
  \ex The kids ran \p{around} in the kitchen.
\end{xlist}\end{exe}

See also: \psst{Instrument}, \psst{Manner}

\begin{history}
  The v1 hierarchy distinguished many different subcategories of path descriptions. 
  The labels \sst{Traversed}, \sst{1DTrajectory}, \sst{2DArea}, \sst{3DMedium}, 
  \sst{Contour}, \sst{Via}, \sst{Transit}, and \sst{Course} have all been merged
  with \psst{Path} for v2.
\end{history}

