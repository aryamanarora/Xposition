\section{Circumstance}

\shortdef{Macrolabel for labels pertaining to space and time,
and other relations that are usually semantically non-core properties of events.}

\psst{Circumstance} is used directly for:
\begin{itemize}
  \item \textbf{Contextualization}
\begin{exe}
    \ex \p*{In}{in} arguing for tax reform, the president claimed that loopholes allow 
    big corporations to profit from moving their headquarters overseas.
    \ex\label{ex:in-activity-Circumstance} 
      You crossed the line \p{in} sharing confidential information.\\{} 
      [but see \cref{ex:in-activity-Topic} under \psst{Topic}, which is syntactically parallel]
    \ex I found out \p{in} our conversation that she speaks 5~languages.
    \ex\rf{Circumstance}{Locus}:\begin{xlist}
      \ex I haven't seen them \p{in} that setting.
      \ex \p*{In}{in} that case, I wouldn't worry about it.
    \end{xlist}
    \ex We have to keep going \p{through} all these challenges. [metaphoric motion] (\rf{Circumstance}{Path})
    \ex Bipartisan compromise is unlikely \p{with} the election just around the corner.
    \ex\label{ex:as-while} \p*{As}{as} we watched, she transformed into a cat. 
    [`while', `unfolding at the same time as'; not simply providing a `when'---contrast~\cref{ex:as-when} under \psst{Time}]
  \end{exe}
  For these cases, the preposition helps situate 
  the background context in which the main event takes place. 
  The background context is often realized as a subordinate clause 
  preceding the main clause. 
  It may also be realized as an adjective complement:
  \begin{exe}
    \ex\begin{xlist}
      \ex My tutor was helpful \p{in} giving concrete examples and exercises.
      \ex You were correct \p{in} \choices{answering the question\\your answer}.
    \end{xlist}
  \end{exe}
  Relatedly, we use \psst{Circumstance} to analyze \pex{involved \p{in}}:
  \begin{exe}
    \ex\begin{xlist}
      \ex I was involved \p{in} a car accident. (\psst{Circumstance})
      \ex Many steps are involved \p{in} the process of buying a home. \\(\rf{Whole}{Circumstance})
    \end{xlist}
  \end{exe}
  \item \textbf{Setting events}
  \begin{exe}
    \ex\label{ex:settingevt} We are having fun \choices{\p{at} the party\\\p{on} vacation}. (\rf{Circumstance}{Locus})
  \end{exe}
  The object of the preposition is a noun denoting a containing event; 
  it thus may help establish the place, time, and/or reason for the governing scene, 
  but is not specifically providing any one of these, despite the locative preposition.
  These can be questioned (at least in some contexts) with \emph{Where?} or \emph{When?}. 
  \Cref{ex:settingevt} entails \cref{ex:settingevtpred}:
  \begin{exe}
    \ex\label{ex:settingevtpred} We are \choices{\p{at} the party\\\p{on} vacation}. (\rf{Circumstance}{Locus})
  \end{exe}
  which may be responsive to the questions \emph{Where are you?} and \emph{What are you doing?}.\footnote{When 
  the object of the preposition is not a (dynamic) event, as with \pex{We are \p{at} odds/\p{on} medication}, 
  \rf{Manner}{Locus} is used: see discussion of state PPs at \psst{Manner}.}
  
  \item \textbf{Occasions}
  \begin{exe}
    \ex I bought her a bike \p{for} Christmas.
    \ex I had peanut butter \p{for} lunch.
  \end{exe}
  These simultaneously express a \psst{Time} and some element of causality 
  similar to \psst{Purpose}.
  But the PP is not exactly answering a \pex{Why?}\ or \pex{When?}\ question. 
  Instead, the sentence most naturally answers a question like \pex{On what occasion was X done?}
  or \pex{Under what circumstances did X happen?}.
  \item Any other descriptions of event/state properties that are \textbf{insufficiently specified} 
  to fall under spatial, temporal, causal, or other subtypes like \psst{Manner}. E.g.:
  \begin{exe}
    \ex\label{ex:over-lunch} Let's discuss the matter \p{over} lunch. [compare \cref{ex:at-lunch}]
  \end{exe}

\item \textbf{Conditions}
\begin{exe}
  \ex You can leave \choices{\p{as\_long\_as}\\provided} your work is done.
  \ex Whether you can leave \choices{depends \p{on}\\is subject \p{to}} whether your work is done.
\end{exe}
\end{itemize}

