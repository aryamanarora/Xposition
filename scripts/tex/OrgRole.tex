\section{OrgRole}

\shortdef{Either party in a relation between an 
organization\slash institution and an individual who has a stable affiliation 
with that organization, such as membership or a business relationship.}

Like its supertype \psst{SocialRel}, \psst{OrgRole} 
lacks any prototypical adposition, but participates in numerous construals:

\begin{exe}
  \ex \rf{OrgRole}{Gestalt} with the institution as possessor:
  {\setlength\multicolsep{0pt}%
  \begin{multicols}{2}
    \begin{xlist}
      \ex the chairman \p{of} the board
      \ex the president \p{of} France
      \ex \choices{employees\\customers} \p{of} Grunnings
      
      \sn the board\p{'s} chairman
      \sn France\p{'s} president
      \sn Grunnings\p{'s} \choices{employees\\customers}
    \end{xlist}
  \end{multicols}}
  \ex \rf{OrgRole}{Gestalt} with possessive marking on the individual:
    \begin{xlist}
      \ex \p{my} school/gym [that I attend]
      \ex \p{my} work [the place where I work]
      \ex \p{my} landscaping company [that I hired]
    \end{xlist}
  \ex \rf{OrgRole}{Possessor} if the individual is understood to possess authority 
    within or as a representative of the institution:
    \begin{xlist}
      \ex \p{my} small business [that I own or operate]
      \ex the president\p{'s} administration
    \end{xlist}
  \ex\begin{xlist}
    \ex Mr. Dursley works \p{for} Grunnings. (\rf{OrgRole}{Beneficiary})
    \ex Mr. Dursley works \p{at} Grunnings. (\rf{OrgRole}{Locus})
    \ex Mr. Dursley is \p{from} Grunnings. (\rf{OrgRole}{Source})
    \ex Mr. Dursley is \p{with} Grunnings. (\rf{OrgRole}{Accompanier})
    \ex Mr. Dursley is employed \p{by} Grunnings. (\rf{OrgRole}{Agent}) %\nss{or do we say `employ' is just a regular Agent/Theme verb?}
  \end{xlist}
  \ex I always do business \p{with} this company. (\rf{OrgRole}{Co-Agent})
  \ex\rf{OrgRole}{Accompanier}:\begin{xlist}
    \ex I bank \p{with} TSB.
    \ex my phone service \p{with} Verizon
  \end{xlist}
  \ex For my Honda I always got replacement parts \p{through} the dealership. [intermediary business] (\rf{OrgRole}{Instrument})
  \ex I serve \p{on} the committee. (\rf{OrgRole}{Locus})
  \ex\rf{OrgRole}{Stuff} if the governor is an organizational collective noun 
  and the object of the preposition describes the full membership:
  	\begin{xlist}
      \ex An order \p{of} nuns
      \ex	A chamber group \choices{\p{of}\\\p{with}} 5 players
    \end{xlist}
  \ex\rf{OrgRole}{Characteristic} if the governor is an organizational collective noun 
  and the object of the preposition denotes a subset of members:
  	\begin{xlist}
      \ex	A piano quintet is a chamber group \p{with} a piano (in it)
    \end{xlist}
\end{exe}

A family counts as an institution 
construed as a \psst{Whole} (set of its members) 
or as a \psst{Locus}:
\begin{exe}
  \ex I am the baby \p{of} the family. (\rf{OrgRole}{Whole})
  \ex people \p{in} my family (\rf{OrgRole}{Locus})
\end{exe}

For a relation between a unit and a larger institution, 
use \psst{Whole}:
\begin{exe}
  \ex the Principals Committee \p{of} the National Security Council (\psst{Whole})
\end{exe}

See also: \psst{Stuff}

\begin{history}
  \psst{OrgRole} is now distinguished within the broader \psst{SocialRel} category 
  following the precedent of the Abstract Meaning Representation \citep[AMR;][]{amr,amr-guidelines}. 
  In AMR, \texttt{have-org-role-91} captures relations between 
  an individual and an institution (such as an organization or family),
  whereas \texttt{have-rel-role-91} is used for relations between two individuals.
\end{history}

