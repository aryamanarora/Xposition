\section{Quantity}

\shortdef{Something measured by a quantity denoted by the governor.}

The governor may be a precise or vague count/measurement. 
This includes nouns like ``lack'', ``dearth'', ``shortage'', ``excess'', or ``surplus''
(meaning a too-small or too-large amount).

Question test: the governor answers ``How much/many of (object)?''

The main preposition is \p{of}.

\begin{itemize}
\item Simple \psst{Quantity}:
\begin{exe}
  \ex\label{ex:bottleQuantity}	Pour me a bottle('s worth) \p{of} beer. [but see \cref{ex:bottleStuff}]
  \ex	I have 2 years \p{of} training.
  \ex	\begin{xlist}
    \ex I ate \choices{6 ounces\\a piece} \p{of} cake.
    \ex	An ounce \p{of} compassion
  \end{xlist}
  \ex	There's a dearth \p{of} cake in the house.
  \ex	This cake has thousands \p{of} sprinkles.
  \ex They number in the tens \p{of} thousands.
  \ex	\begin{xlist}
    \ex\label{ex:anumber} I have a \choices{number\\handful} \p{of} students.
    \ex	I have a lot \p{of} students.
    \ex	We did a lot \p{of} traveling.
    \ex	There is a lot \p{of} wet sand on the beach.
  \end{xlist}
  \ex	A pair \p{of} shoes
\end{exe}

\item If the measure includes a word like ``amount'', ``quantity'', or ``number'',\footnote{Excluding 
the expression ``a number'' meaning `several', as in \cref{ex:anumber}.} 
the construal \rf{Quantity}{Gestalt} is used 
(because the amount of something can be viewed as an attribute):
\begin{exe}
  \ex\label{ex:QuantityGestalt} \rf{Quantity}{Gestalt}:
  \begin{xlist}
    \ex	A generous amount \p{of} time
    \ex A large number \p{of} students
  \end{xlist}
\end{exe}
But if ``amount'', ``quantity'', etc. is used without a measure as its modifier, 
it is simply \psst{Gestalt}: see \cref{ex:amountGestalt}.

\item If the governor is a \textbf{collective noun} not denoting an organization, 
the construal \rf{Quantity}{Stuff} is used 
(note that a ``consisting of'' paraphrase is possible):
\begin{exe}
  \ex\label{ex:QuantityStuff} \rf{Quantity}{Stuff}:
  \begin{xlist}
    \ex Can you outrun a herd \p{of} wildebeest?
    \ex Put 3 bales \p{of} hay on the truck.
    \ex	\choices{A group\\2 groups\\A throng} \p{of} vacationers just arrived.
  \end{xlist}
\end{exe}
For organizational collectives, see \psst{OrgRole}.

\item Otherwise, if the object refers to \textbf{a specific item or set}, 
and the quantity measures a portion of that item 
(whether a quantifier, absolute measure, or fractional measure),
the construal \rf{Quantity}{Whole} is used:
\begin{exe}
  \ex\label{ex:QuantityWhole} \rf{Quantity}{Whole}:
  \begin{xlist}
    \ex	I ate 6 ounces \p{of} the cake in the refrigerator.
    \ex	I ate \choices{half\\50\%} \p{of} the cake.
    \ex	\choices{All/many/lots/a lot/\\some/few/both/none} \p{of} the town's residents 
    are students.
    \ex	I have seen all \p{of} the city. (= the whole city)
    \ex	A lot \p{of} the sand on the beach is wet.
    \ex	2 \p{of} the children are redheads.
    \ex 2 \p{of} the 10 children in the class are redheads.
  \end{xlist}
\end{exe}
However, simple \psst{Whole} is used if the portion is specified as 
``the rest'', ``the remainder'', etc., as in \cref{ex:rest}. %\nss{Reconsidering this: see \cref{ex:rest}}\ab{I see "half of it", "remainder of it" do have some shared meaning. Why the three "5 ounces of it", "half of it" and "remainder of it" seem different is that in the 1st case, the quantity of the proportion itself is specified, for the 2nd case, it is unspecified but can be determined from knowledge about the quantity of the whole thing and in the 3rd case, it is unspecified and cannot be %determined from knowing the quantity of the whole thing only, you also need to know how much has already been removed.}
\end{itemize}

