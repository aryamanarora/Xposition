\section{Beneficiary}

\shortdef{Animate or personified undergoer that is (potentially) 
advantaged or disadvantaged by the event or state.}

This label does not distinguish the polarity of the relation 
(helping or hurting, which is sometimes termed \emph{maleficiary}).

\begin{exe}
  \ex Vote \choices{\p{for}\\\p{against}} Pedro!
  \ex Junk food is bad \p{for} your health.
  \ex My parrot died \p{on} me.
  \ex\begin{xlist} 
    \ex These are clothes \p{for} children.
    \ex These are children\p{'s} clothes. (\rf{Beneficiary}{Possessor})
  \end{xlist}
  \ex Fortunately \p{for} the turkey('s future), he received a presidential pardon.
\end{exe}

\noindent Specific subclasses include:
\begin{itemize}
\item	Animate who will potentially experience a benefit or harm as a result of something 
but is not an experiencer or recipient of the main predicate itself. 
(May be an experiencer or recipient of the result.)
\item	Animate target of emotion or behavior, discussed below. %(rude TO women, compassion FOR animals)
\item	Animate who someone supports or opposes (e.g., \emph{vote \p{for}}, 
\emph{cheer \p{for}}, \emph{Hooray \p{for}}). % but "Well done TO John" is Recipient ~> Goal)
\item Intended user/usee: 
  \begin{exe}
    \ex (We sell) clothes \p{for} children
    \ex a gallows \p{for} criminals
    \ex This is the car \p{for} you! [advertising idiom]
  \end{exe}
\item	Something characterized as good/appropriate (or not) for some kind of 
\textbf{animate user or usee}, delimiting the applicability of a descriptor 
to that kind of individual: 
  \begin{exe}
    \ex \begin{xlist}
      \ex This place is great \p{for} young children.
      \ex This is a great place \p{for} young children.
    \end{xlist}
  \end{exe}
\end{itemize}
The first and last items above have analogues with \psst{Purpose}. 
The key difference is that \psst{Beneficiary} applies to an animate participant, 
whereas \psst{Purpose} applies to an intended consequence or one of its inanimate participants.

\paragraph{Targets of behavior versus emotion.}
A preposition can mark an individual in the context of evaluating how someone else is treating them, 
with a noun or adjective governor. 
If behavior is more salient than emotion, then \psst{Beneficiary} is the scene role. 
If emotion is highly salient, then \psst{Stimulus} is the scene role.
\begin{exe}
  \ex Behavior-focused:
    \begin{xlist}
      \ex She exhibits rudeness \p{towards} customers. (\rf{Beneficiary}{Direction})
      \ex He is \choices{rude\\condescending} \p{to} women. (\rf{Beneficiary}{Goal})
      \ex He is gentle and compassionate \p{with} animals. (\rf{Beneficiary}{Theme})
    \end{xlist}
  \ex Emotion-focused, repeated from \cref{ex:StimBen}:
    \begin{xlist} % Also seems to be a metaphorical connection to Direction, but that would require multiple construal to express.
      \ex Her disdain \p{for} customers was apparent. (\rf{Stimulus}{Beneficiary})
      \ex He has/feels compassion \choices{\p{towards}\\\p{for}} animals. (\rf{Stimulus}{Beneficiary})
    \end{xlist}
\end{exe}
Note that the emotion-focused examples can describe private emotional states directly, 
while the behavior-focused examples are behavior-based judgments or inferences about emotional states.

An obligation directed at somebody is analyzed like targeted behavior:
\begin{exe}
  \ex We have a solemn responsibility \p{to} our armed forces. (\rf{Beneficiary}{Goal})
\end{exe}

Similar to the behavior-focused examples, inanimate causes can have the potential to positively or negatively
affect somebody. Ability and permission modalities are included here:
\begin{exe}
    \ex\begin{xlist}
      \ex The strategy is \choices{beneficial \\ risky \\ an option} \p{for} investors. (\psst{Beneficiary})
      \ex The strategy \choices{is helpful \\ poses a risk \\ is available} \p{to} investors. (\rf{Beneficiary}{Goal})
    \end{xlist}
\end{exe}

\paragraph{Versus \psst{Recipient}.}
\psst{Beneficiary} applies to the classic English benefactive construction 
where it is ambiguous between assistance and intended-transfer:
\begin{exe}
  \ex John baked a cake \p{for} Mary. [to help Mary out, and/or with the intention of giving her the cake]
\end{exe}
However, if transfer (or communication) is the main semantics of the scene 
and benefit or harm is no more than an inference, then the scene role is \psst{Recipient}:
\begin{exe}
  \ex a \choices{message\\gift} \p{for} my mother (\rf{Recipient}{Direction})
  \ex a \choices{package} \p{for} the front office (\rf{Recipient}{Direction})
\end{exe}

See also: \psst{Experiencer}, \psst{OrgRole}

