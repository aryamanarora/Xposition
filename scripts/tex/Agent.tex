\section{Agent}

\shortdef{Animate instigator of an action (typically volitional).}

%Prototypical prepositions are \p{by} (prominently including passive-\p{by}), \p{of}, and \p{'s}:
This is most directly associated with the passive \p{by}-phrase (\cref{sec:passives}), 
but also permits other construals:
\begin{exe}
  \ex the decisive vote \p{by} the City Council
  \ex\rf{Agent}{Gestalt}:\begin{xlist}
    \ex the decisive vote \p{of} the City Council
    \ex the City Council\p{'s} decisive vote
    \ex they needed Joan\p{'s} help
    \ex It was \choices{the chairman\p{'s} fault\\the fault \p{of} the chairman}.
  \end{xlist}
  %\ex ?they needed help of hers
  %(\rf{Agent}{Possessor})
\end{exe}
When two symmetric \psst{Agent}s are collected in a single NP 
functioning as a set, it is marked as a \psst{Whole} construal:
\begin{exe}
  \ex There was a war \p{between} France and Spain. (\rf{Agent}{Whole})
  \ex a discussion \p{among} the board members (\rf{Agent}{Whole})
%  \ex Please talk \p{amongst} yourselves. \rf{Agent}{Whole}
% reflexives are weird. maybe Co-Agent

\end{exe}

Compare: \psst{Co-Agent}; 
see also: \psst{OrgRole}, \psst{Originator}, \psst{Source}, \psst{Stimulus}


