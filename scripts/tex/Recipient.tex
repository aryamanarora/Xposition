\section{Recipient}

\shortdef{The party (usually animate) that is the endpoint of (actual or intended) transfer of a thing or message, 
becoming the final \psst{Possessor} or \psst{Gestalt}.
Excludes events where transfer/communication is not framed as unidirectional.}
A ``goal'' in the broadest sense of an ending point/condition. 
Contrasts with \psst{Originator}.

English construals:\footnote{If subject position is viewed as an \psst{Agent} construal, 
then active subject with a transfer verbs like \pex{get} or \pex{receive} is \rf{Recipient}{Agent}.
If direct object position is viewed as a \psst{Theme} construal, 
then \pex{She informed \uline{her editor}} are \rf{Recipient}{Theme}.}
\begin{exe}
  \ex She \choices{gave the story\\spoke} \p{to} her editor. (\rf{Recipient}{Goal})
  \ex What title did you give \p{to} your essay? [inanimate] (\rf{Recipient}{Goal})
  \ex news \p{for} our readers (\rf{Recipient}{Direction})
  \ex He is yelling \p{at} me to get ready! (\rf{Recipient}{Direction}\footnote{While \emph{yell \p{at}} 
often has a connotation of shouting criticism towards somebody, 
and criticism would suggest \psst{Beneficiary},
the \psst{Recipient} aspect of the meaning is more explicit and essential:
yelling from a distance at someone does not imply criticism, 
and criticism about someone who is absent is not yelling at them.})
  \ex The news was not well received \p{by} the White House. (\rf{Recipient}{Agent})
  \ex Timmy\p{'s} piano lesson (\rf{Recipient}{Gestalt})
  \ex I'll have to check \p{with} my supervisor. (\rf{Recipient}{Co-Agent})
\end{exe}

\psst{Recipient} does not apply to events like \pex{exchange/talk/chat (\p{with})}, 
which involve a back-and-forth between 
\psst{Agent} and \psst{Co-Agent} (or a plural \psst{Agent} subject):
\begin{exe}
  \ex She \choices{swapped stories\\chatted} \p{with} her friends. (\psst{Co-Agent})
\end{exe}

See also: \psst{Beneficiary}

\begin{history}
  In v1, \psst{Recipient} was the counterpart to \sst{Donor/Speaker}:
  \psst{Recipient} was a subtype of \sst{Destination}, which 
  inherited from \sst{Location} and \psst{Goal}. 
  Moving \psst{Recipient} directly under \psst{Participant} 
  puts it in a neutral position with respect to its possible construals.
\end{history}

