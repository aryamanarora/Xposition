\section{Explanation}

\shortdef{Assertion of \textbf{why} something happens or is the case.}

This marks a secondary event that is asserted as the reason for the main event or state. 

\begin{exe}
\ex  I went outside \p{because\_of} the smell. %(Why did you go outside?)
\ex  The rain is \p{due\_to} a cold front. %(Why is there rain?)
\ex  He reacted \p{out\_of} anger. (\rf{Explanation}{Source})
\ex\begin{xlist}
  \ex He thanked her \p{for} the cookies.
  \ex Thank you \p{for} being so helpful.
\end{xlist}
\end{exe}

When a preposition like \p{after} is used and the relation is temporal as well as causal, 
construal captures the overlap. While \p{since} and \p{as} can also be temporal, 
there are tokens where they cannot be paraphrased respectively with \p{after} and \emph{when}:
\begin{exe}
  \ex I joined a protest \p{after} the shameful vote in Congress. (\rf{Explanation}{Time})
  \ex Her popularity has grown \p{since} she announced a bid for president. (\rf{Explanation}{Time})
  \ex I will appoint him \choices{\p{since}\\\p{as}\\\#\p{after}\\\#when} he is most qualified for the job. (\psst{Explanation})
\end{exe}



Question test: \psst{Explanation} and its subtype \psst{Purpose} license
\pex{Why?} questions.

