\section{Manner}

\shortdef{The style in which an event unfolds, the form that something takes, 
or the condition that something is in.}

% FN definition of Manner under Expend_resource: 
%Any description of the intentional act which is not covered by more specific FEs, 
%including secondary effects (quietly, loudly), and general descriptions comparing events 
%(the same way). In addition, it may indicate salient characteristics of an Agent 
%that also affect the action ( deliberately, eagerly, carefully). 

\psst{Manner} is used as the scene role for several kinds of descriptors 
which typically license some sort of \pex{How?} question:

\begin{itemize}
  \item The style in which an action is performed or an event unfolds, 
  expressed adverbially (canonical use of the term ``manner''):
  \begin{exe}
    \ex He reacted \choices{\p{with} anger\\\p{in} anger\\angrily}.\footnote{\pex{He reacted \p{out\_of} anger} is \rf{Explanation}{Source}.}
    \ex He reacted \p{with} nervous laughter. [contrast: \psst{Means}]
%His reaction was angry; His angry reaction: also Manner, though not prepositional
    \ex I made the decision \choices{\p{by} myself\\\p{without} anyone else\\\p{on}\_~~my~~\_own}. [see \cref{sec:refl}]
    \ex \rf{Manner}{ComparisonRef}:\begin{xlist}
      \ex You eat \p{like} a pig (eats).
      \ex You smell \p{like} a pig.
    \end{xlist}
    \ex\label{ex:smellOf} \choices{Your father smells\\The soup tastes} \p{of} elderberries. (\rf{Manner}{Stuff}) [also~\cref{ex:smellOfNotCmp}]
  \end{exe}
  
  \item An adverbial \textbf{depictive} characterizing a participant of an event:
  \begin{exe}
    \ex She entered the room \choices{\p{in} a stupor\\drunk}. (\rf{Manner}{Locus})
  \end{exe}

  \item The \textbf{form or shape} that something takes, including language of communication 
  and shape of motion:
  \begin{exe}
    \ex\label{ex:in-shape} \rf{Manner}{Locus}:\begin{xlist}
      \ex The clothes are (sitting) \p{in} a pile. [contrast adnominal use: \cref{ex:in-pile-adnominal} under \psst{Whole}]
      \ex The ribbon is (tied) \p{in} a bow.
      \ex The sand is \p{in} a pyramid shape.
      \ex\label{ex:pathmanner} They dance \p{in} a circle.
      \ex I read the book \p{in} French.
      \ex The book is \p{in} French.
      \ex music \p{in} C major
      \ex She loves teaching, and it shows \p{in} her smile.
    \end{xlist}
  \end{exe}
  
  \item \emph{What} + \p{like} (\emph{what he looks \p{like}}, etc.): see \cref{ex:whatlike} under \psst{ComparisonRef}.
  
  \item The \textbf{state or condition} that something is in. 
  The PP or intransitive preposition is used (especially predicatively) 
  to describe a qualitative state or condition of something (especially an entity) 
  that is not simply a relation of location, time, possession, quantity, causation, etc.\ 
  between governor and object.
  For example:
  
  \begin{itemize}
    \item With the noun \pex{state}, \pex{condition}, etc.:
    
    \begin{exe}
      \ex\rf{Manner}{Locus}:\begin{xlist}
        \ex The chairs are \p{in} excellent shape.
        \ex I'm \p{in} no condition to go outside.
      \end{xlist}
    \end{exe}

    \item Bodily/medical conditions presented as applying to the governor:
    
    \begin{exe}
      \ex John is \choices{\p{on} his back\\\p{on} antibiotics\\\p{on} the ventilator\\\p{in} pain\\\p{in} a coma}. (\rf{Manner}{Locus})
    \end{exe}
    
    \item Miscellaneous qualitative senses of specific prepositions used statively:
    
    \begin{exe}
      \ex John is \choices{\p{for}\\\p{against}} the war. [opinion] (\rf{Manner}{Beneficiary})
      \ex John is \p{into} sports. [hobbies/interests] (\rf{Manner}{Goal})
    \end{exe}
%Contrast with cases where the object is presented as a partial property of the entity-denoting governor: “The girl with pain / with a cold” is Characteristic

    \item Idiomatic PPs expressing states, for example:\footnote{Often the object of the preposition 
    is determinerless (\pex{\p{in} business}) \citep{baldwin-06} 
    or has a fixed determiner (\pex{\p{in} a hurry}).}
    \begin{exe}
    \ex \pex{\p{on} fire} (contrast \pex{\p{in} the fire}), 
      \pex{\p{on} time} (contrast \pex{\p{at} the time}), 
      \pex{\p{in} trouble}, \pex{\p{in} love}, \pex{\p{in} tune}, \pex{\p{in} a hurry}, 
      \pex{\p{at} odds}, \pex{\p{out\_of} business}, \pex{\p{out\_of} control} (\rf{Manner}{Locus})
    \ex They are \choices{\p{on}\_~~the~~\_way\\\p{on}\_~~their$_{\text{\backposs}}$~~\_way}  (\rf{Manner}{Locus})
    \end{exe}
    
    \item Intransitive prepositions expressing a qualitative state (not location, time, etc.):
    \begin{exe}
      \ex\rf{Manner}{Locus}:\begin{xlist}
        \ex The lights are \p{off}.
        \ex The party tonight is \p{on}. [scheduled to happen; not canceled]
        \ex Political TV shows are \p{in}. [in fashion]
      \end{xlist}
    \end{exe}
  \end{itemize}
  
  A few observations about these state PPs are in order.
  
  \begin{enumerate}
  \item In a reversal of the usual asymmetry between governor and adpositional object, 
  semantically, the PP defines the kind of scene that the governor participates in. 
  To an extent, this may be true of all predicative PPs, but the state PPs are often such 
  that the object of the preposition is neither an event nor a referential entity. 
  I.e., \pex{John is \p{in} a hurry} does not exactly express a relation 
  between the entities \pex{John} and \pex{a hurry}; rather, it expresses something 
  qualitative about the entity \pex{John}'s condition.
  
  \item The most idiomatic of the state PPs seem to resist questions of the form 
  \emph{What?}+NP-supercategory with a stranded preposition:
  \begin{exe}
    \ex More productive prepositional usages:\begin{xlist}
      \ex The party is \p{in} January.~$\rightarrow$ What month is the party \p{in}? [Or: When is the party?] (\psst{Time})
      \ex John is \p{on} aspirin.~$\rightarrow$ What medication is John \p{on}?\footnote{Or, colloquially, with a suspected mind-altering substance: \pex{What is John \emph{\p{on}}?!}} (\rf{Manner}{Locus})
    \end{xlist}
    \ex\label{ex:idiomPP} Less productive/more idiomatic preposition + NP combinations:\begin{xlist}
      \ex John is \p{in} \choices{a hurry\\a coma}.~$\nrightarrow$ What \_ is John \p{in}?\footnote{\pex{What condition/state is John \p{in}?} does work, but is quite vague.}  (\rf{Manner}{Locus})
      \ex John is \p{on} fire.~$\nrightarrow$ What \_ is John \p{on}? (\rf{Manner}{Locus})
    \end{xlist}
  \end{exe}

  \item Typically these states are binary: something is either \pex{\p{on} fire}\slash \pex{\p{on} time}, or not. 
  For some, the negation may be expressed by substituting a contrasting preposition: 
  an orchestra that is not \pex{\p{in} tune} is \pex{\p{out\_of} tune}.
  \end{enumerate}

  \paragraph{State PPs with complements.}
  The \rf{Manner}{Locus} construal is also used when there is effectively a preposition+NP+preposition  
  combination that links two arguments:
  \begin{exe}
    \ex\rf{Manner}{Locus}:\begin{xlist}
      \ex\label{ex:InLove} John is \p{in} love (with$_{\text{\rf{Stimulus}{Topic}}}$ Mary). [cf.~\cref{ex:InLoveWith}]
      \ex That is \p{at} odds with$_{\text{\rf{ComparisonRef}{Topic}}}$ our agreement.
    \end{xlist}
  \end{exe}
  
  \paragraph{Idiomatic PP with modifier slot: \pex{on a(n)\dots basis}.}
  There seems to be a construction \pex{\p{on} a(n) MODIFIER basis} where the 
  modifier phrase reflects the scene role being filled. 
  We use \psst{Manner} as the function:
  \begin{xexe}
    \ex\label{ex:bipartisanBasis} The legislation was passed \p{on}\_a\_~~bipartisan~~\_basis. (\psst{Manner})
    \ex I see them \p{on}\_a\_~~daily~~\_basis. (\rf{Frequency}{Manner}) [also \cref{ex:dailyBasis}]
  \end{xexe}

  \paragraph{Change-of-state PPs.}
  Occasionally, a PP will mark a start or end state, in which case we collapse 
  the state/location distinction, using \psst{Source} or \psst{Goal} as the scene role:
  \begin{exe}
    \ex John came \p{out\_of} a coma. (\psst{Source})
    \ex John slipped \p{into} a coma. (\psst{Goal})
    \ex The drugs put John \p{in} a coma. (\rf{Goal}{Locus})
    \ex They chopped the wood \p{in} pieces. (\rf{Goal}{Locus})
  \end{exe}
  



    
  \end{itemize}
  
  %\ex You can use it \p{as} a hammer. (\rf{Manner}{Identity}\nss{?})

% \nss{TODO: separate section about predicative PPs:}
% Other highly specialized preposition usages in free combination with objects
% We are out of eggs (Possession)
% But: We are rolling in dough; Romney was born in(to) money/wealth; Married into money; ?We are in money
% Maybe ‘wealth’ is interpreted as a state and that was extended to ‘money’/’dough’
% Married into the newspaper business
% The plane was on time; The plane arrived on time / in a timely fashion; The plane’s arrival was on time / timely; The plane’s on-time / timely arrival: all Manner ~> Locus. (Though the state expressed by the PP involves time, it does not directly answer a When? question, so it is not labeled Time.)
% Intransitive: The lights are off
% While somewhat borderline, we deem out of town as Locus as it answers a Where? question.\nss{noted under PP idioms}

\paragraph{Versus \psst{Circumstance}.}
State PPs like \pex{\p{at} odds} and \pex{\p{on} medication}, which receive the 
construal \rf{Manner}{Locus}, are similar to situating events like 
\pex{\p{at} the party} and \pex{\p{on} vacation}, which are analyzed as 
\rf{Circumstance}{Locus}. What matters for the scene role is whether the object 
of the preposition is an event or not.

\paragraph{Versus \psst{Characteristic}.} 
Note that, despite the prevailing use of the term `manner' for descriptors of \emph{events}, 
\psst{Manner} and \psst{Characteristic} each cover certain descriptors of \emph{entities}.
The use of \rf{Manner}{Locus} to cover the state that an entity is in was deemed necessary 
to account for the dual function of PPs like \emph{\p{in} French} as
predicate complements (\emph{the book is \p{in} French}) and 
adverbials (\emph{read it \p{in} French}).


When an entity is being described, 
it is difficult to semantically separate \textbf{states} from \textbf{attributes}. 
We currently make the distinction based on whether there is a construal of \emph{partiality}, 
using \psst{Characteristic} or its subtype \psst{Possession}
for partial-attributes like possessions, attire, and body part conditions
(\emph{the man \p{with} a hat}; \emph{He is \p{in} a suit}; \emph{the man \p{with} an ear infection}). 
\psst{Manner} is reserved for conditions where the object of the preposition cannot be 
localized with respect to the entity 
(\emph{the man \p{with} a cold}; \emph{He is \p{in} pain}).
However, the feasibility of this distinction may be worth revisiting in the future.\futureversion{\nss{One possibility
to consider for v3: merge Characteristic and Manner as Quality,
and treat the partiality difference as a matter of construal with different functions.
E.g. Quality construed as State/Predicate vs. Quality construed as Component.}}

\paragraph{Versus \psst{ComparisonRef}.}
See \psst{ComparisonRef}.

\begin{history}
  In v1, \psst{Manner} was positioned as an ancestor of all 
  categories that license a \emph{How?} question, including 
  \psst{Instrument}, \psst{Means}, and \sst{Contour}, as in \cref{ex:pathmanner}. 
  This criterion was deemed too broad, so \psst{Manner} has no 
  subtypes in v2.
\end{history}

