\section{OTHER INFINITIVE (\backi)}\label{sec:specialinf}

As described in \cref{sec:inf}, infinitive clauses are analyzed with a supersense 
if and only if they serve as a purpose adjunct, or in certain purpose-related constructions 
(inherent purpose, action that costs money in a commercial scene, 
that which something is sufficient or excessive for).
The special label \backi is reserved for all other uses of infinitival \p{to}, 
as well as \p{for} whenever it introduces the subject of an infinitive clause.\footnote{Essentially, 
our position is that these uses of infinitivals are more like syntactically core elements 
(subject, object) than obliques, and thus should be excluded from semantic annotation 
under the present scheme.}

Infinitivals warranting \backi include:
\begin{exe}\ex\begin{xlist}
  \ex I want \p{to} meet you. [complement of control verb]
  \ex I would\_like \p{to} try the fish. [\pex{would\_like} is a polite alternative to \pex{want}]
  \ex It seems \p{to} be broken. [complement of raising verb]
  \ex You have an opportunity \p{to} succeed. [complement of noun]
  \ex I'm ready \p{to} leave. [complement of adjective]
  \ex I'm glad \p{to} hear you're engaged! [complement of emotion adjective]
  \ex You're great/a pleasure \p{to} work with. [complement of evaluative adjective or noun]
  \ex These new keys are expensive \p{to} copy. [tough-movement]
  \ex My plan is \p{to} eat at noon. [infinitival as NP]
  \ex It's impossible \p{to} get an appointment. [infinitival as NP, with cleft]
  \ex I know how \p{to} lead. [complement of wh-word]
  \ex I have nothing \p{to} hide. [complement of indefinite pronoun]
  \ex Do you have time \p{to} help me? [with resource, not necessity]
  \ex They took\_the\_time \p{to} listen to my concerns. [complement of verbal idiom]
\end{xlist}\end{exe}

Multiword auxiliaries---such as quasi-modals \pex{have\_to} `must', \pex{ought\_to} `should', etc., 
as well as \pex{have\_yet\_to}---subsume the infinitival \p{to}, so no label on \p{to} is required:
\begin{exe}
  \ex You have\_to choose a date.
\end{exe}

Whenever \p{for} introduces a subject of an infinitival clause, the \p{for} token is labeled 
\backi (regardless of whether \p{to} receives a semantic label; see \cref{sec:inf}):
\begin{exe}\ex\begin{xlist}
  \ex I need [\p{for}$_{\text{\backi}}$ you \p{to}$_{\text{\backi}}$ help me].
  \ex I opened the door [\p{for}$_{\text{\backi}}$ Steve \p{to}$_{\psst{Purpose}}$ take out the trash].
\end{xlist}\end{exe}

