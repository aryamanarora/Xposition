\section{Infinitive Clauses}\label{sec:inf}

In its function as infinitive marker, \p{to} is not generally considered to be a preposition. 
Nevertheless, we consider all uses of \p{to} for adposition supersense annotation 
because infinitive clauses (infinitivals) can express similar semantic relations 
as prepositional phrases. Most notably, infinitival purpose adjuncts alternate 
with \p{for}-PP purpose adjuncts:
\begin{exe}
  \ex\label{ex:infPurpose}\psst{Purpose}:\begin{xlist}
    \ex\begin{xlist}
      \ex Open the door \p{to} let in some air.
      \ex Open the door \p{for} some air.
    \end{xlist}
    \ex\begin{xlist}
      \ex I flew to headquarters \p{to} meet with the principals.
      \ex I flew to headquarters \p{for} a meeting with the principals.
    \end{xlist}
  \end{xlist}
\end{exe}
Thus, from a practical point of view, we might as well treat infinitival \p{to} 
as capable of marking a \psst{Purpose}.

The following is an exhaustive list of semantic analyses that we consider for infinitivals: 
\begin{itemize}
  \item \textbf{Purpose adjuncts}, generally adverbial, as in \cref{ex:infPurpose}. 
  These are labeled \psst{Purpose}. They can generally be paraphrased with \p{in\_order\_to}.
  
  \item \textbf{Inherent purposes}, generally adnominal, as in \cref{ex:charPurp}, 
  described under \psst{Purpose}. 
  These are labeled \rf{Characteristic}{Purpose}.
  
  \item In a \textbf{commercial scene}, that which costs money; labeled \rf{Theme}{Purpose}.
  Repeated from the discussion under \psst{Theme}:
  \begin{exe}
    \ex\begin{xlist}
      \ex They asked \$500 \p{to} make the repairs. (\rf{Theme}{Purpose})
      \ex \$500 \p{to} make the repairs was excessive. (\rf{Theme}{Purpose})
    \end{xlist}
  \end{exe}
  
  \item Constructions of \textbf{sufficiency and excess}---\pex{too short \p{to} ride}, 
  \pex{not tall enough \p{to} ride}, etc., where the assertion of sufficiency or excess 
  licenses an infinitival, labeled \rf{ComparisonRef}{Purpose}. 
  See discussion at \psst{ComparisonRef}.
\end{itemize}

Infinitival tokens not covered by this list are labeled \backi (\cref{sec:specialinf}).

\paragraph{Infinitival with \p{for}-subject.}
In \cref{ex:infPurpose}, the infinitive clause has no local subject---rather, 
an argument of the matrix clause doubles as the subject of the infinitive clause 
(control). However, a separate subject can be introduced with \p{for}, 
in which case \p{for}+NP is treated as a dependent of the infinitive verb 
and labeled \backi:
\begin{exe}
  \ex\begin{xlist}
    \ex I opened the door [\p{for}$_{\text{\backi}}$ Steve \p{to}$_{\psst{Purpose}}$ take out the trash].
    \ex It cost \$500 [\p{for}$_{\text{\backi}}$ the mechanic \p{to}$_{\text{\rf{Theme}{Purpose}}}$ make the repairs].
  \end{xlist}
\end{exe}

% 