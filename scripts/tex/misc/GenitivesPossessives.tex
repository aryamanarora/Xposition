\section{Genitives/Possessives}\label{sec:genitives}

\Citet{blodgett-18} detail the application of this scheme to English possessive constructions:
the so-called \textbf{s-genitive}, as in \cref{ex:SGen}, and 
\textbf{of-genitive}, as in \cref{ex:OfGen}:
\begin{exe}
  \ex\label{ex:SGen} 
    \begin{xlist}
      \ex \choices{the Smith family\p{'s}\\\p{their}} house (\psst{Possessor})
      \ex \choices{the tea\p{'s}\\\p{its}} price (\psst{Gestalt})
    \end{xlist}
  \ex\label{ex:OfGen} 
    \begin{xlist}
      \ex the house \p{of} the Smith family (\psst{Possessor})
      \ex the price \p{of} the tea (\psst{Gestalt})
    \end{xlist}
\end{exe}
Note that the s-genitive is realized with case marking (clitic \p{'s} or possessive pronoun\footnote{For ease of indexing, 
\p{'s} or \p{s'} is preferred over possessive pronouns for s-genitive examples in this document.}) 
rather than a preposition, 
and the case-marked NP in the s-genitive alternates with the object of the preposition in the of-genitive.
(This may feel unintuitive: annotators looking at the s-genitive construction are often tempted to focus on 
the role occupied by the head noun rather than the case-marked noun.)

The s-genitive and of-genitive are particularly associated with 
\psst{Possessor} (which applies to a canonical form of possession) 
and the more general category \psst{Gestalt}; both supersenses are illustrated above \cref{ex:SGen,ex:OfGen}.
In addition, both genitive constructions can mark participant roles and other kinds of relations, 
including \psst{Whole} and \psst{SocialRel} relations. 
When the s-genitive is used, the \emph{function} is always either \psst{Gestalt} (most cases) 
or \psst{Possessor} (when the possession is sufficiently canonical).
While overlapping in scene roles with the s-genitive, 
\p{of} is considered compatible with some additional functions, 
including \psst{Whole}, \psst{Source}, and \psst{Theme}; thus of-genitives 
with such roles do not need to be construed as \psst{Gestalt} or \psst{Possessor}:
\begin{exe}
  \ex\rf{SocialRel}{Gestalt}:\begin{xlist}
    \ex the grandfather \p{of} Lord Voldemort
    \ex \choices{Lord Voldemort\p{'s}\\\p{his}} grandfather
  \end{xlist}
  \ex\begin{xlist}
    \ex the hood \p{of} the car (\psst{Whole})
    \ex the nose \p{of} He-Who-Must-Not-Be-Named (\psst{Whole})
    \ex \choices{the car\p{'s} hood\\\p{its}} (\rf{Whole}{Gestalt})
    \ex \choices{He-Who-Must-Not-Be-Named\p{'s} nose\\\p{his}} (\rf{Whole}{Gestalt})
  \end{xlist}
  \ex\begin{xlist}
    \ex the arrival \p{of} the queen (\psst{Theme})
    \ex \choices{the queen\p{'s} arrival\\\p{her}} (\rf{Theme}{Gestalt})
  \end{xlist}
  \ex \choices{Shakespeare\p{'s}\\\p{his}} works (\rf{Originator}{Gestalt})
  \ex These are children\p{'s} clothes.\footnote{Cannot readily be paraphrased with \p{their} because \w{children} is not referential, 
  but rather refers to a kind. This construction has been termed the \emph{descriptive genitive} \citep[pp.~322, 327--328]{quirk-85}.} [clothes intended for use and possession by children] (\rf{Beneficiary}{Possessor})
\end{exe}

The literature on the genitive alternation examines the factors that condition 
the choice of construction; important factors include the length and animacy of the possessed NP.
In addition, \p{of} participates in certain constructions that are not really possessives---%
e.g.~\pex{this sort \p{of} sweater} (\psst{Species}).

Certain idioms require an s-genitive argument that does not participate in 
any transparent semantic relationship; for these, \backposs is used (\cref{sec:possidiom}).

%\subsubsection{Construct State Genitive in Hebrew}

