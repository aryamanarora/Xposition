\section{DISCOURSE (\backd)}\label{sec:discourse}

Discourse connectives and other markers that transition between ideas 
or convey speaker attitude/hedging/emphasis/attribution but do not belong 
to propositional content. Examples include:

\begin{exe}
\ex \p{according\_to}; \w{\p{after}\_all}, \w{\p{of}\_course}, \w{\p{by}\_the\_way}; 
\w{\p{for}\_chrissake} (interjection); 
\w{\p{above}\_all}, \w{\p{to}\_boot}; % \w{more\_often\_than\_not}, <-- not prepositional as a whole
\w{\p{in}\_other\_words}, \w{\p{on}\_the\_other\_hand}; 
\w{\p{in} my experience}, \w{\p{in}\_my\_opinion}
\end{exe}

This label also covers ``additive focusing markers'' 
\citep[p.~592]{cgel} with a meaning similar to `also' or `too',
where an item is added to something already established in the discourse:
\begin{exe}
  \ex\begin{xlist}
    \ex I shot the sheriff \p{as}\_well.
    \ex They serve coffee, and tea \p{as}\_well.
  \end{xlist}
\end{exe}
It also covers topicalization markers:
\begin{exe}
  \ex \p*{As\_for}{as\_for} the sheriff, well, I shot 'im.
\end{exe}
Finally, \backd applies to adpositions relating a metalinguistic mention of 
a speech act to the speech content itself---whether the adposition 
introduces this speech act mention, as in \cref{ex:toSumItUp},
or links the discourse expression to a subordinate statement, as in \cref{ex:sumItUpWith}.
\begin{exe}
  \ex\begin{xlist}
    \ex\label{ex:toSumItUp} \p*{To}{to} sum it up: It was a terrible experience.
    \ex\label{ex:sumItUpWith} I will sum it up \p{with}: It was a terrible experience.
  \end{xlist}
\end{exe}

