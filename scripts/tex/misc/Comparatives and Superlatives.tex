\section{Comparatives and Superlatives}

Various constructions express a comparison between two arguments. 

\paragraph{\psst{ComparisonRef} for second argument.}
When the second argument (the point of reference) 
is adpositionally marked, \psst{ComparisonRef} is used, regardless of 
its complement's syntactic type: 
\begin{exe}
  \ex\label{ex:comparisonrefArg}\begin{xlist} 
        \ex Your face is as red \p{as} \choices{a rose\\mine is}. (\psst{ComparisonRef})
        \ex Your face is redder \p{than} \choices{a rose\\mine is}. (\psst{ComparisonRef})
    \end{xlist}
\end{exe}
See further examples at \psst{ComparisonRef}.

\subsubsection{\p*{As}{as}-\p{as} comparative construction}\label{sec:as-as}

\paragraph{\psst{Extent} argument.} 
In an \p{as}-\p{as} comparison, the scene role of the first argument 
(the object of the first \p{as}) is the role that would be operative 
if the construction were removed and only the first argument remained: 
e.g., \pex{I stayed as long as I could} $\rightarrow$ \pex{I stayed long}.
The function of the first \p{as} is always \psst{Extent} 
to reflect that it marks the degree on a scale:
\begin{exe}
  \ex\begin{xlist}
    \ex I helped \p{as} much as I could. (\psst{Extent})
    \ex Your face is \p{as} red as a rose. (\rf{Characteristic}{Extent})
    \ex I helped \p{as} carefully as I could. (\rf{Manner}{Extent})
    \ex I stayed \p{as} long as I could. (\rf{Duration}{Extent})
    \ex I helped \p{as} often as I could. (\rf{Frequency}{Extent})
    \ex I've eaten (twice) \p{as} much (food) as you. [amount of something]\\ 
    (\rf{Approximator}{Extent})
  \end{xlist}
\end{exe}

\paragraph{Second argument: \psst{ComparisonRef}.} 
See \cref{ex:comparisonrefArg} above.

\subsubsection{Superlatives}\label{sec:superlative}

\psst{Whole} is used for the superset or gestalt licensed by a superlative:
\begin{exe}
  \ex the youngest \p{of} the children (\psst{Whole})
\end{exe}
See more at \psst{Whole}.


