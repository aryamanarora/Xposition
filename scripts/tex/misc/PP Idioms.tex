\section{PP Idioms}

Many PPs exhibit some amount of lexicalization or idiomaticity.
This is especially true of PPs that tend to be used predicatively.
In general it is extremely difficult to establish tests to distinguish idiomatic PPs 
from fully productive combinations. 
However, the usual criteria apply for the supersense analysis.

For example, if the PP answers a \emph{Where?}\ question, 
it qualifies as \psst{Locus}; 
whereas qualitative states usually have \psst{Manner} as the scene role:
\begin{exe}
  \ex He is \p{out\_of} town. (\psst{Locus})
  \ex The company is \p{out\_of} business. (\rf{Manner}{Locus})
\end{exe}
See further discussion at \psst{Manner}.

\subsubsection{Reflexive PP Idioms}\label{sec:refl}

Certain idiomatic constructions involve a preposition that requires a reflexive 
direct object.

\paragraph{PERFORM-ACTIVITY \emph{for} oneself.}
\begin{itemize}
  \item When something is done for one's own benefit rather than someone else's:
    \begin{exe}
      \ex I took a vacation \p{for} myself (\psst{Beneficiary})
    \end{exe}
  \item When something is done in a way that affords direct rather than second-hand information:
    \begin{exe}
      \ex You should try out the restaurant \p{for} yourself! (\rf{Agent}{Beneficiary})
    \end{exe}
\end{itemize}
\paragraph{PERFORM-ACTIVITY \emph{by} oneself.}
\begin{itemize}
  \item When something is done without accompaniment (the negation would be \emph{\p{with} others}):
    \begin{exe}
      \ex I had lunch (all) \p{by} myself [`alone'] (\psst{Accompanier}\footnote{Though \emph{myself} 
      is not literally accompanying \emph{I}, the PP as a whole describes the nature of accompaniment (or lack thereof).}) 
    \end{exe}
  \item When something is accomplished without assistance:
    \begin{exe}
      \ex I made the decision (all) \p{by} myself. (\psst{Manner})
      \ex The computer rebooted all \p{by} itself. (\psst{Manner})
    \end{exe}
\end{itemize}
\paragraph{BE \emph{by} oneself.}
Alone; unaccompanied:
\begin{exe}
  \ex I am \p{by} myself right now. (\psst{Accompanier})
\end{exe}

