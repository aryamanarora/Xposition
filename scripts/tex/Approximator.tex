\section{Approximator}

\shortdef{An ``operator'' that semantically takes a measurement, 
quantity, or range as an argument and ``transforms'' it in some way 
into a new measurement, quantity, or range.}

For instance:
\begin{exe}
  \ex We have \p{about} 3 eggs left.
  \ex We have \p{in\_the\_vicinity\_of} 3 eggs left.
  \ex We have \p{over} 3 eggs left.
  \ex We have \p{between} 3 and 6 eggs left.
\end{exe}
Similarly for \p{around}, \p{under}, \p{more\_than}, \p{less\_than}, \p{greater\_than}, 
\p{fewer\_than}, \p{at\_least}, and \p{at\_most}.\footnote{These constructions are 
markedly different from most PPs; it is even questionable whether these usages 
should count as prepositions. Without getting into the details here, 
even if their syntactic status is in doubt, 
we deem it practical to assign them with a semantic label in our inventory because they 
overlap lexically with ``true'' prepositions.}

