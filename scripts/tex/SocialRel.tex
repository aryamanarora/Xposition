\section{SocialRel}

\shortdef{Entity, such as an institution or another individual, 
with which an individual has a stable affiliation.}

Typically, \psst{SocialRel} applies directly to interpersonal relations 
(versus the subtype \psst{OrgRole} for relations involving an organization).
It does not have any prototypical adpositions. 
Construals include:
\begin{exe}
  \ex \begin{xlist}
      \ex\label{ex:workwithSR} I work \p{with} Michael. (\rf{SocialRel}{Co-Agent})
      \ex Joan has a class \p{with} Miss Zarves. (\rf{SocialRel}{Co-Agent})
    \end{xlist}
  \ex\rf{SocialRel}{Gestalt} 
  {\setlength\multicolsep{0pt}%
  \begin{multicols}{2}
    \begin{xlist}
      \ex Joan is the \choices{sister\\wife} \p{of} John.
      \ex Joan is a student \p{of} Miss Zarves.
      \ex the family \p{of} Miss Zarves
      
      \sn Joan is John\p{'s} \choices{sister\\wife}.
      \sn Joan is Miss Zarves\p{'s} student.
      \sn Miss Zarves\p{'s} family 
    \end{xlist}
  \end{multicols}}
  \ex Joan is studying \p{under} Prof.~Smith. (\rf{SocialRel}{Locus})
  \ex Joan is married \p{to} John. (\rf{SocialRel}{Co-Theme})
  \ex Joan is divorced \p{from} John. (\rf{SocialRel}{Co-Theme})
  \ex Joan bought her house \p{through} a real estate agent. [intermediary] (\rf{SocialRel}{Instrument})
\end{exe}

Note, however, that \emph{work \p{with}} is ambiguous between 
being in an established professional relationship \cref{ex:workwithSR}, 
and engaging temporarily in a joint productive activity:
\begin{exe}
  \ex\label{ex:workwithCA} I was working \p{with} Michael after lunch. (\psst{Co-Agent})
\end{exe}
It is up to annotators to decide from context which interpretation 
better fits the context.

\begin{history}
  Renamed from v1 label \sst{ProfessionalAspect}, which was borrowed from 
  \citet{srikumar-13,srikumar-13-inventory}.
  The name \psst{SocialRel} reflects
  a broader set of stative relations involving an individual 
  in a social context, including kinship and friendship.
  See also note under \psst{OrgRole}.
\end{history}

