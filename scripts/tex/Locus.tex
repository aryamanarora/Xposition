\section{Locus}

\shortdef{Location, condition, or value. May be abstract.}

\begin{exe}
  \ex I like to sing \choices{\p{at} the gym\\\p{on} Main St.\\\p{in} the shower}.
  \ex The cat is \choices{\p{on\_top\_of}\\\p{off}\\\p{beside}\\\p{near}} the dog.
  \ex There are flowers \choices{\p{between}\\\p{among}} the trees.
  %\ex the wheels \p{on} the bus % Whole ~> Locus
  \ex\label{ex:onRight} When you drive north, the river is \p{on} the right.
  \ex I read it \choices{\p{in} a book\\\p{on} a website}.
  \ex The charge is \p{on} my credit card.
  \ex We met \p{on} a trip to Paris.
  \ex The Dow is \p{at} \choices{a new high\\20,000}.
  %\ex I am now \p{off} work.\nss{\rf{Circumstance}{Locus}?}
  \ex That's \p{in} my price range.
  %\ex She was \p{in} a coma.\nss{\rf{Circumstance}{Locus}?}
\end{exe}
The \psst{Locus} may be a part of another scene argument:
part of a figure whose static orientation is described, 
or a focal part of a ground where contact with the figure occurs:\footnote{\psst{PartPortion} 
was considered but rejected for these cases. Instead we assume the verb 
semantics would stipulate that it licenses a \psst{Theme} as well as a (core) \psst{Locus} 
which must be a part of that \psst{Theme}.}
\begin{exe}
  \ex She was lying \p{on} her back.
  \ex\begin{xlist}
    \ex She kissed me \p{on} the cheek.
    \ex I want to punch you \p{in} the face.
  \end{xlist}
\end{exe}
Words that incorporate a kind of reference point are \psst{Locus} 
even without an overt object:
\begin{exe}
  \ex\begin{xlist}
    \ex The cat is \p{inside} the house.
    \ex The cat is \p{inside}.
  \end{xlist}
  \ex\begin{xlist}
    \ex All passengers are \p{aboard} the ship.
    \ex All passengers are \p{aboard}.
  \end{xlist}
\end{exe}
\psst{Locus} also applies to \p{in}, \p{out}, \p{off}, \p{away}, \p{back}, 
etc.\ when used to describe a location without an overt object:
\begin{exe}
  \ex\begin{xlist}
    \ex The doctor is \choices{\p{in}\\\p{out\_of}\\\p{away\_from}} the office.
    \ex The doctor is \choices{\p{in}\\\p{out}\\\p{away}}.
    \ex They are \p{out} to eat.
  \end{xlist}
\end{exe}
And to \p{around} meaning `nearby' or `in the area':
\begin{exe}
  \ex Will you be \p{around} in the afternoon?
  \ex She's the best doctor \p{around}!
\end{exe}

In a phenomenon called \textbf{fictive motion} \citep{talmy-96}, 
dynamic language may be used to describe static scenes. 
We use construal for these:
\begin{exe}
  \ex A road runs \p{through} my property. (\rf{Locus}{Path})
  \ex John saw Mary \choices{\p{through} the window\\\p{over} the fence}.\footnote{The scene establishes a static spatial arrangement of John, Mary, and the window\slash fence, 
  with only metaphorical motion. Yet this is a non-prototypical \psst{Locus}: it cannot be questioned with \emph{Where?}, for example. 
  Moreover, we understand from the scene that the object of the preposition is something with respect to which the viewer is navigating in order to see without obstruction.} (\rf{Locus}{Path})
  \ex The road extends \p{to} the river. (\rf{Locus}{Goal})
  \ex I saw him \p{from} the roof. (\rf{Locus}{Source})
  \ex\label{ex:protesters} Protesters were \choices{kept\\missing} \p{from} the area. (\rf{Locus}{Source})
  \ex\begin{xlist} 
    \ex We live \p{across\_from} you. (\rf{Locus}{Source})
    \ex We're just \p{across} the street from$_{\text{\rf{Locus}{Source}}}$ you. (\rf{Locus}{Path}) %\nss{what construal for My house is across\_from yours? cf. away\_from}
  \end{xlist}
\end{exe}
Construal is also used for prepositions licensed by scalar adjectives of distance, 
\cref{ex:adjDist}, and prepositions used with a cardinal direction, \cref{ex:cardinal}:
\begin{exe}
  \ex\label{ex:adjDist} \begin{xlist}
    \ex We are quite close \p{to} the river. (\rf{Locus}{Goal})
    \ex We are quite far \p{from} the river. (\rf{Locus}{Source})
  \end{xlist}
  \ex\label{ex:cardinal} \begin{xlist}
    \ex The river is \p{to} the north. (\rf{Locus}{Goal}) [cf.~\cref{ex:onRight}]
    \ex The river is north \p{of} Paris. (\rf{Locus}{Source})
  \end{xlist}
\end{exe}
See also \rf{Locus}{Direction} for static distance measurements, 
described under \psst{Direction}.

Qualitative states are analyzed as \rf{Manner}{Locus}, as described under \psst{Manner}.


