\section{Source}

\shortdef{Initial location, condition, or value. May be abstract.}

For motion events, the initial location is where the thing in motion 
(the figure) starts out.
\psst{Source} also applies to abstract or metaphoric initial locations, 
including initial states in a dynamic event.
%and simple \psst{Source} is used for adpositions that generically mark starting points.

In English, a prototypical \psst{Source} preposition is \p{from}:
\begin{exe}
  \ex\label{ex:catBox} The cat jumped \choices{\p{from}\\\p{out\_of}} the box.
  \ex\label{ex:catLedge} The cat jumped \choices{\p{from}\\\p{off\_of}\\\p{off}} the ledge.
  \ex\label{ex:internet} I got it \choices{\p{from}\\\p{off}} the internet.
  \ex people \p{from} France
  \ex The temperature is rising \p{from} a low of 30 degrees.
  \ex I have arrived \p{from} work.
  \ex We discovered he was French \p{from} his attire. [indication]
  \ex I made it \p{out\_of} clay. [material]
  \ex\label{ex:coma} She \choices{awoke \p{from}\\came \p{out\_of}} a coma.
  \ex We are moving \p{off\_of} that strategy.
\end{exe}
The \psst{Source} use of \p{from} can combine with a specific locative PP:
\begin{exe}
  \ex I took the cat \p{from} behind$_{\psst{Locus}}$ the couch.
\end{exe}
Note that \p{away\_from} is ambiguous between marking a starting point (\psst{Source}) 
and a separate orientational reference point (\psst{Direction}):
\begin{exe}
  \ex At the sound of the gun, the sprinters ran \choices{\p{away\_from}\\\p{from}} the starting line. (\psst{Source})
  \ex The bikers ride parallel to the river for several miles, then 
  head east, \choices{\p{away\_from}\\\#\p{from}} the river. 
  (\psst{Direction}: bikers are never at the river)
\end{exe}
%
%Other prepositions specify more information about the nature of the figure's 
%position and trajectory vis-\`{a}-vis its initial location: 
%e.g., \p{off} and \p{off\_of} for motion away from the top of a surface, 
%and \p{out\_of} for motion through the boundary of a container. 
%We say this involves construal of a \psst{Source} as a \psst{Direction}:
% \begin{exe}
%   \exp{ex:catBox} The cat jumped \p{out\_of} the box. (\rf{Source}{Direction})
%   \exp{ex:catLedge} The cat jumped  the ledge. (\rf{Source}{Direction})
% \end{exe}
% \nss{possible objection: we don't distinguish \p{at} (generic \psst{Locus}) 
% vs. \p{on} and \p{in} (surface and container, resp.). So why distinguish their \psst{Source} counterparts?}
%
% If figurative motion language is entrenched and bleached such that the 
% image of a specific kind of motion is no longer salient, simple \psst{Source} is used. 
% The construal as \psst{Direction} is maintained only if the figurative path
% is somewhat salient:
%
Note, too, that \p{off(\_of)} and \p{out(\_of)} can also mark simple states:
\begin{exe}
  \ex I am \p{off} \choices{medications\\work}. (\rf{Manner}{Locus})
  \ex The lights are \choices{\p{off}\\\p{out}}. (\rf{Manner}{Locus})
  \ex Stay \p{out\_of} trouble. (\rf{Manner}{Locus})
\end{exe}
States are discussed at length under \psst{Manner}. 
There is also a (negated) possession sense of \p{out}/\p{out\_of}:
\begin{exe}
  \ex We are \p{out\_of} toilet paper. (\psst{Possession})
\end{exe}

Sometimes a specific \psst{Source} is implicit, and the preposition is intransitive. 
But if no specific referent is implied, another label may be more appropriate:
\begin{exe}
  \ex The cat was sitting on the ledge, then jumped \p{off}. (\psst{Source}: implicit `(of) it')
  \ex He was offered the deal, but walked \p{away}. (\psst{Source}: implicit `from it')
  \ex The bird flew \choices{\p{away}\\\p{off}}. (\psst{Direction}: vaguely away from the viewpoint)
\end{exe}

\psst{Source} is prototypically inanimate, 
though it can be used to construe animate \psst{Participant}s 
(especially \psst{Originator} and \psst{Causer}).
Contrasts with \psst{Goal}.

\paragraph{Agency as giving.}
When an \psst{Agent}'s action to help somebody is conceptualized as 
giving, and the nominalized action as the thing given, 
then \p{from} can mark the \psst{Agent} (metaphorical giver).
If the \p{from}-PP is adnominal, \rf{Agent}{Source} is used \cref{ex:AgentSource}.
However, if the \p{from}-PP is adverbial, and the verb relates to the metaphoric 
transfer rather than the event described by the action nominal, 
then the argument linking becomes too complicated for this scheme to express; 
simple \psst{Source} is used by default \cref{ex:AgentiveSource}:
\begin{exe}
  \ex\label{ex:AgentSource} The attention \p{from} the staff made us feel welcome. (\rf{Agent}{Source})
  \ex\label{ex:AgentiveSource}\psst{Source}:\begin{xlist} 
    \ex I received great care \p{from} this doctor.
    \ex I got a second chance \p{from} her.
    \ex I need a favor \p{from} you.
  \end{xlist}
\end{exe}

