\section{Identity}

\shortdef{A category being ascribed to something, 
or something belonging to the category denoted by the governor.}

Prototypical prepositions are \p{of} (where the governor is the category) 
and \p{as} (where the object is the category):
\begin{exe}
  \ex\label{ex:stateof} the state \p{of} Washington [as opposed to the city]
  \ex The liberal state \p{of} Washington has not been receptive to Trump's message.
  \ex \p*{As}{as} a liberal state, Washington has not been receptive to Trump's message.
  \ex\label{ex:ascolleague} I like Bob \p{as} a colleague. [but not as a friend]
  \ex What a gem \p{of} a restaurant! [exclamative idiom: both NPs are indefinite]
  \ex the problem/task/hassle \p{of} raising money
  \ex the age \p{of} eight
  \ex They did a great job \p{of} cleaning my windows.
  \ex\label{ex:shell} \rf{Topic}{Identity}, with a governing noun in the domain of communication or cognition:
    \begin{xlist}
      \ex the topic/issue/question \p{of} semantics
      \ex the idea \p{of} raising money
    \end{xlist}
\end{exe}
Something may be specified with a category in order to disambiguate it \cref{ex:stateof}, 
or to provide an interpretation or frame of reference with which that entity is to be considered.
In some cases, like \cref{ex:shell}, the category is a \emph{shell noun} \citep{schmid-00} 
requiring further specification.

Categorizations may be situational rather than permanent/definitional:
\begin{exe}\ex\label{ex:assituational}\begin{xlist}
  \ex She appears \p{as} Ophelia in \emph{Hamlet}.
  \ex He is usually a bartender, but today he is working \p{as} a waiter.
\end{xlist}\end{exe}

Paraphrase test: ``(thing) IS (category) [in the context of the event]'': 
``Washington is a liberal state'', ``opening a new business is a hassle'', 
``She is Ophelia'', etc. Note that \p{as}+category may attach syntactically 
to a verb, as in \cref{ex:ascolleague} and  \cref{ex:assituational}, 
rather than being governed by the item it describes.

If the object of the preposition is a property (as opposed to a category), 
the scene role is \psst{Characteristic}:
\begin{exe} 
  \ex Adnominal: \rf{Characteristic}{Identity}\begin{xlist}
    \ex a car \p{of} high quality
    \ex a man \p{of} honor
    \ex a business \p{of} that sort [contrast with \psst{Species}, \cref{sec:Species}]
  \end{xlist}
  \ex Secondary predicate adjective: \rf{Characteristic}{Identity}\begin{xlist}
    \ex She described him \p{as} sad.
    \ex He strikes me \p{as} sad.
  \end{xlist}
\end{exe}

See also: \psst{ComparisonRef}

\begin{history}
  Generalized from v1, where it was called \sst{Instance} and restricted 
  to the ``(category) \p{of} (thing)'' formulation. 
  The relevant usages of \p{as} were labeled \sst{Attribute}.
\end{history}

