\section{Originator}

\shortdef{Animate who is the initial possessor or creator/producer of something,
including the speaker/communicator of information. 
Excludes events where transfer/communication is not framed as unidirectional.}

A ``source'' in the broadest sense of a starting point/condition. 
Contrasts with \psst{Recipient} if there is transfer/communication.

English construals:\footnote{If we consider subject position as an \psst{Agent} construal 
and direct object position as a \psst{Theme} construal, then we can add examples like 
\pex{\uline{She} talked to her editor} (\rf{Originator}{Agent}) and 
\pex{They robbed \uline{her} of her life savings} (\rf{Originator}{Theme}).
\psst{Originator} does not apply to the subject of events like \pex{exchange} or \pex{talk/chat (with)}, 
which involve a back-and-forth between 
\psst{Agent} and \psst{Co-Agent} (or a plural \psst{Agent}).}
\begin{exe}
  \ex \rf{Originator}{Agent} (passive-\p{by} or adnominal \p{by}):
  \begin{xlist}
    \ex\label{ex:worksBy} works \p{by} Shakespeare [cf.~\cref{ex:worksOf,ex:worksGen}]
    \ex The telephone was invented \p{by} Alexander Graham Bell.
    \ex The story was \choices{given\\told} to$_{\text{\rf{Recipient}{Goal}}}$ her \p{by} her editor.
  \end{xlist}
  \ex \rf{Originator}{Source}:
  \begin{xlist}
    \ex\label{ex:worksOf} works \p{of} Shakespeare [cf.~\cref{ex:worksBy,ex:worksGen}]
    \ex The story was obtained \p{from} an anonymous White House employee.
    \ex I bought it \p{from} this company.
    \ex I heard the news \p{from} Larry.
  \end{xlist}
  \ex \rf{Originator}{Gestalt}: \begin{xlist}
    \ex \label{ex:worksGen} Shakespeare\p{'s} works [cf.~\cref{ex:worksOf,ex:worksBy}]
    \ex Rodin\p{'s} sculptures
    \ex the store\p{'s} fresh produce
    \ex the restaurant\p{'s} food
    \ex John\p{'s} \choices{question\\speech}
  \end{xlist}
\end{exe}

\paragraph{\emph{learn from.}} If the source of learning is an individual 
(or group of individuals, organization, etc.)\ who provides information, 
\rf{Originator}{Source} applies. Otherwise, it is simply \psst{Source}:
\begin{exe}
  \ex We learned a lot \p{from} Miss Zarves. (\rf{Originator}{Source})
  \ex We learned a lot \p{from} that \choices{book\\experience}. (\psst{Source})
\end{exe}


% \nss{Do we need to limit further the kinds of events that have an Originator 
% and Recipient? Only communicative events where the message is core? 
% (``Say'', ``tell'', ``inform'', but not ``talk'' or ``negotiate''. 
% Otherwise we have ``talk with him'' as \rf{Originator}{Co-Agent}; 
% ``negotations by the parties'' as \rf{Orignator}{Agent}; 
% and ``negotations between the parties'' as Originator \tat> Agent \tat> Whole!)}

\begin{history}
  \psst{Originator} merges v1 labels \sst{Donor/Speaker} and \sst{Creator}, 
  which were difficult to distinguish in the case of authorship.
  %(e.g., \pex{the operas \p{of} Puccini}).
  \sst{Donor/Speaker} was a subtype of \sst{InitialLocation}, which 
  inherited from \sst{Location} and \psst{Source}. 
  \sst{Creator} was a subtype of \psst{Agent}.
  Moving \psst{Originator} directly under \psst{Participant} 
  puts it in a neutral position with respect to its possible construals.
\end{history}

